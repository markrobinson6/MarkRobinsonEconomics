%%STUFF TO DO:
%%, citations, 
%%add in appendixes
%%does pedro even want the simple model?  read his comments
%see how tables are formtted, does he use makecell?




\documentclass[letter, 12pt, epsf,leqno]{article}
\usepackage{wrapfig}
\usepackage{setspace,caption}
\usepackage{graphics}
\usepackage{graphicx}
\usepackage{amssymb}
\usepackage{amsmath}
\usepackage{theorem}
\usepackage{anysize}
\usepackage[T1]{fontenc}
\usepackage[sc]{mathpazo}
\usepackage{calligra}
\linespread{1.05}   %
\usepackage{epsfig}
\usepackage{subfigure}
\usepackage{appendix}
\usepackage{color,soul}
\usepackage{verbatim}
\usepackage{multirow}
\usepackage{float}
\usepackage[section]{placeins}
\usepackage{booktabs}
\usepackage{threeparttable}
\usepackage{makecell} %added by Mark, not sure if ill use it
%\usepackage{url} %added by Mark, not sure if ill use it
\usepackage[hidelinks]{hyperref}
\usepackage{array}
\usepackage[a4paper,bindingoffset=0.2in,%
            left=1in,right=1in,top=1in,bottom=1in,%
            footskip=.25in]{geometry}

\usepackage{natbib}
\setstretch{1.9}
\usepackage{footmisc}



%\bibliographystyle{ksfh_nat}
%\bibliographystyle{ier}
\bibliographystyle{plainnat}
\setcitestyle{authoryear,open={(},close={)}} %Citation-related commands
%\bibliographystyle{perception}
%\bibliographystyle{plainurl}
%\bibliographystyle{apa}


\renewcommand{\footnotelayout}{\setstretch{1.9}}
\setlength{\footnotesep}{0.7\baselineskip}

\begin{document}
%\title{Labor-Restricting Policies, Driving, and Pollution
\title{When People Work Less, Do They Drive and Pollute More?
% \thanks{Manuscript received April 2018; revised March 2019.}
%\\ \footnotesize SHORTENED TITLE: SOCIAL INSURANCE AND MOBILITY
}

\author{
	%German Cubas\\ 
	%University of Houston, U.S.A.
	%\and
	Mark Robinson\textsuperscript{1}\\ 
        Temple University, U.S.A.}
\maketitle

\begin{abstract}
\noindent
What effect do labor-restricting policies have on how much people drive, and on carbon emissions?  I model the effect of labor-restricting policies in the following way: I calibrate a model to match the United States in 2018, and also calibrate the model counterfactually to simulate what would have occurred had other policies been in effect.  I compare the carbon emissions that result from the original calibration to the carbon emissions that result from the counterfactual-policy calibrations.  The labor-restricting policies I consider are wage taxes, retirement mandates, and restrictions on time spent working.  I find that, for all policies considered, reductions in work are associated with increases in driving but nonetheless lead to reductions in carbon emissions, due to overall declines in economic activity.  





%Does a reduction in labor hours lead to an increase or a decrease in carbon emissions?  I consider two mechanisms.  First, if we hold the emissions released per hour of labor constant, a reduction in labor will lead to a reduction in emissions.  Second, the demand for goods and services changes as people work less; a change to the mix of goods and services being demanded (and produced) may either raise or lower the emissions released per labor-hour.  Since the first mechanism unambiguously reduces carbon emissions, and the second mechanism can point in either direction, the ultimate effect of a reduction in labor on carbon emissions is unclear.In this paper I study an OLG model in which agents can work, engage in leisure, consume automobile services, and consume a general consumption good.  I calibrate the model to match the United States in 2018 and calculate the flow of carbon emissions in the steady-state equilibrium.  I then run several experiments in which policy-variables are changed so that agents are induced to change how much they work; I then compare the flow of carbon emissions in the experimental models to the original model.  I find that, depending on which policy-variables are changed, small reductions in labor can either increase or decrease carbon emissions, but sufficiently large reductions in labor lead to decreases in carbon emissions.\\

\noindent
\textit{Key words:}  Pollution, Leisure, Transportation
Choice \\
\textit{JEL Classifications:} J22 $\cdot$ Q54 $\cdot$ R41.


\end{abstract}





\tableofcontents
\newpage





\footnotetext[1]{{I am grateful to Pedro Silos and many others for valuable input. Please address all correspondence to: Mark 
			Robinson, mark.robinson@temple.edu.}}
\section{Introduction}
\setcounter{footnote}{1}
\indent
Do policies that restrict labor - such as taxing labor, implementing a mandatory retirement age, and restricting how much time can be spent working - lead to a reduction in carbon emissions?  Answering this question requires us to consider how labor-restricting policies affect how much people work, and furthermore if reductions in work lead to reductions in carbon emissions.  

To make this project viable, I focus specifically on emissions caused by driving private vehicles, and how changes in how much people work relate to changes in how much people drive (and pollute).  I build a model in which agents can spend their time working, driving, or at leisure.  The money they earn at work can be used to either pay the monetary cost of driving (gas, car purchase and maintenance, etc.) or it can be spent on a general-consumption good which represents all other goods and services (food, clothing, housing, energy, toys, concerts, etc.)  After calibrating the model to match the United States in 2018, I calculate the flow of carbon emissions when the model is in the steady-state.  I next perform ``experiments'' on the model by calibrating it to simulate counterfactual government policies (such as a tax on wages or mandatory-retirement age) that cause the agents to change how much they work; I then calculate the flow of emissions in the steady-states with these counterfactual calibrations.  By comparing the carbon emissions in the experimental calibrations to the emissions in the original calibration, I can see if reductions in labor are associated with increases or decreases in carbon emissions.

A large body of work exists that estimates the elasticity of the labor supply, the elasticity of driving with respect to the price of fuel and other costs, and the elasticity of driving with respect to income.  Furthermore, researchers have estimated the carbon emissions released by driving.  Together, this large empirical literature - some of which is reviewed below - can shed some light on how policies that restrict work can change how much people drive and how much pollution is released.  The contribution of this paper is to consider all these effects jointly.  Furthermore, this paper's contribution is to develop a model so that we can consider how these different effects interact in equilibrium, rather than considering them in isolation from one another.  

This paper focuses on people's decision to drive private vehicles, as opposed to spending time and money in other ways which are often less-polluting.  Data shows that driving private vehicles is a major source of carbon emissions.  In 2019 about ten percent of global anthropogenic (human-caused) carbon emissions were from road transportation, and between 2010 and 2019, road transport was one of the fastest growing sources of emissions at 1.7 percent per year, although this halted during the COVID-19 pandemic.\footnote{Of the 59 GtCO2eq anthropogenic carbon emissions released globally in 2019 (including emissions released from fossil fuel combusion, industrial processes, and land use changes such as agriculture and forestry), about ten percent originated from road transportation, 5 percent from non-road transport,  16 percent from buildings, 22 percent from agriculture and land use, 34 percent from industry, and 12 percent from other sources (page 2-31 and Figure 2.12 of \citet{ipcc2})}  In the United States in 2020, transport made up 27 percent of emissions, with over half coming from road transport.\footnote{In the US, 25 percent is from electric power, 24 percent from industry, 13 percent from commercial and residential, and 11 percent from agriculture.  "In terms of the overall trend, from 1990 to 2020, total transportation emissions have increased due, in large part, to increased demand for travel. The number of VMT [vehicle miles travelled] by light-duty motor vehicles (passenger cars and light-duty trucks) increased by 30\% from 1990 to 2020, as a result of a confluence of factors including population growth, economic growth, urban sprawl, and periods of low fuel prices." \citep{epa_sources})}


%To make this project tangible, I focus specifically on emissions caused by driving private vehicles, and how changes in how much people work relate to changes in how much people drive (and pollute).  Driving private vehicles is a major source of carbon emissions.  In 2019 about ten percent of global anthropogenic (human-caused) carbon emissions were from road transportation, and between 2010 and 2019, road transport was one of the fastest growing sources of emissions at 1.7 percent per year, although this halted during the COVID-19 pandemic.\footnote{Of the 59 GtCO2eq anthropogenic carbon emissions released globally in 2019 (including emissions released from fossil fuel combusion, industrial processes, and land use changes such as agriculture and forestry), about ten percent originated from road transportation, 5 percent from non-road transport,  16 percent from buildings, 22 percent from agriculture and land use, 34 percent from industry, and 12 percent from other sources (page 2-31 and Figure 2.12 of \citet{ipcc2})}  In the United States in 2020, transport made up 27 percent of emissions, with over half coming from road transport.\footnote{In the US, 25 percent is from electric power, 24 percent from industry, 13 percent from commercial and residential, and 11 percent from agriculture.  "In terms of the overall trend, from 1990 to 2020, total transportation emissions have increased due, in large part, to increased demand for travel. The number of VMT [vehicle miles travelled] by light-duty motor vehicles (passenger cars and light-duty trucks) increased by 30\% from 1990 to 2020, as a result of a confluence of factors including population growth, economic growth, urban sprawl, and periods of low fuel prices." \citep{epa_sources})}  Thus transport makes up a much larger share of US emissions than global emissions.  

%A large body of work exists that estimates the elasticity of the labor supply, the elasticity of driving with respect to the price of fuel and other costs, and the elasticity of driving with respect to income.  Furthermore, researchers have estimated the carbon emissions released by driving.  Together, this large empirical literature - some of which is reviewed below - can shed some light on how policies that restrict work can change how much pollution is released.  The contribution of this paper is to consider all these effects jointly.  Furthermore, this paper's contribution is to develop a model so that we can consider how these different effects interact in equilibrium, rather than considering them in isolation from one another.  





%Driving private vehicles is a major source of carbon emissions, whether we look at international or national data.  According to the IPCC's Sixth Assessment Report, in 2019 there were about 59 GtCO2eq anthropogenic (human-caused) carbon emissions.  That number includes emissions relating from fossil fuel combustion and industrial processes, as well as land use changes such as agriculture and forestry.  Of the 59 GtCO2eq, about ten percent originated from road transportation, making it a large source of emissions.  About 16 percent of emissions come from buildings, 22 percent from agriculture and land use, and 34 percent from industry, 5 percent from non-road transport, and 12 percent from other sources (page 2-31 of \citet{ipcc2}) Between 2010-2019, road transport was one of the fastest growing sources of emissions at 1.7 percent of the year, although this halted during the COVID-19 pandemic (page 10-10 of \citet{ipcc10}).
% \footnote{page 10-10 of \url{www.ipcc.ch/report/ar6/wg3/downloads/report/IPCC_AR6_WGIII_Chapter_02.pdf}, page 2-31 and 2-19 of \url{www.ipcc.ch/report/ar6/wg3/downloads/report/IPCC_AR6_WGIII_Chapter_02.pdf}}
%\footnote{Note that in these numbers, energy and electricity are not counted as a separate sector because they are counted with the sector that uses them.  But however electricity and heat are counted, transport is about fifteen percent of total emissions.  See Figure 2.12 of \citet{ipcc2})}

%The United States EPA looks specifically at emissions from the United States in 2020 and says that transport makes up 27 percent of emissions, with over half coming from road transport.  25 percent is from electric power, 24 percent is from industry, 13 percent from commercial and residential, and 11 percent from agriculture.  Thus transport makes up a much larger share of US emissions than global emissions.  US emissions was growing from 1991 to 2004, in part because of the increasing popularity of fuel efficient vehicles.  While that trend has since waned, the rise of VMT has driven the trend of rising emissions from transportation.  "In terms of the overall trend, from 1990 to 2020, total transportation emissions have increased due, in large part, to increased demand for travel. The number of VMT by light-duty motor vehicles (passenger cars and light-duty trucks) increased by 30\% from 1990 to 2020, as a result of a confluence of factors including population growth, economic growth, urban sprawl, and periods of low fuel prices." \citep{epa_sources})  After 2004, US transportation emissions roughly stayed level, dipping during the Great Recession and then almost recovering afterward, until they steeply declined at the beginning of the pandemic.


In this paper, I focus on two mechanisms by which a reduction in labor can affect the flow of carbon emissions.

First, if the amount of carbon-emissions per hour of labor stays constant, then a reduction in labor will lead to a reduction in carbon emissions.  This mechanism is perhaps the simplest to describe: people work less, so less goods and services are produced and consumed, so the pollution associated with production and consumption decreases.  This is the ``reduced production'' effect, although it could easily be called the ``reduced consumption'' effect.

Second, as people work less, there may be a reallocation of production (that is, of how much of each kind of good or service is produced).  Working less may cause people to demand a different mix of goods and services.  For example, consider two goods at the opposite ends of the pollution spectrum: airplane flights, which are high-polluting, and meditation classes, which are low-polluting.  If people who work less demand more flights, the economy may produce more emissions per labor-hour.  But if a reduction in labor hours leads to an increased demand for (and production of) meditation classes, the economy may produce less emissions per labor-hour.  This is the ``reallocation-of-production'' effect.

If we consider only the reduced-production mechanism, a reduction in labor unambiguously leads to a reduction in emissions.  But if we consider only the reallocation-of-production mechanism - or we consider both mechanisms acting jointly - a reduction in labor could lead to either a reduction or an increase in emissions.\footnote{
In order to best focus on these two mechanisms, this paper mainly ignores other possible mechanisms through which a reduction in labor could affect emissions, such as changes in technology or production techniques.  These mechanisms are discussed in the Conclusion.
}

To study the reallocation-of-production mechanism, we need to know if people working less leads to them demanding more or less-polluting goods.  As mentioned earlier, I focus specifically on driving private vehicles, as private vehicles are an important example of a good that causes much pollution.  If people drive much more when they work less, one can imagine that a reduction in labor would cause little or no reduction in carbon emissions, or perhaps even increase it (because the reallocation-of-production mechanism would dominate the reduced-production mechanism).  But if people drive about the same relative to other goods and services when they work less, this would suggest that work-reductions lead to emission-reductions.\footnote{Excluding trips whose purpose is to go home, 25.3 percent of trips are for work and 23 percent are for either meals or social/recreation, 29.7 percent for shopping and errands, 4.7 for school or daycare (4.7), 2.3 for medical and dental, 12.9 for transporting someone else, and 2.1 for other purposes.  See \citet{transportation}.  We should be cautious in how we use these numbers to predict how a reduction in work hours would affect vehicle miles driven, in part because the data refers to to the purpose of trips rather than the purpose of miles.  Furthermore, it is difficult to ascertain the true ``cause'' of any particular trip because trips can be chained together (for example, a person using the car to run errands on the way home from work).  A reduction in work hours could change the number and times a person commutes to the workplace, which could cause the person to re-shuffle their use of the car in myriad ways.} % \footnote{Much research on trip chaining exists}



In this paper, I run ``experiments'' on a calibrated model by having the government (in the model) implement tax changes and other policies that induce agents to change how much they work.  As the agents change how much they work, they also change how much they drive, consume, save, and pollute.  Thus, the model can be used to estimate how policies which change how much people work affect pollution levels.\footnote{
The idea that a reduction in economic activity can lead to a reduction in pollution is well-illustrated by the lockdowns in the early part of the pandemic.  \citet{liu} found ``an abrupt 8.8\% decrease in global CO2 emissions (-1551 Mt CO2) in the first half of 2020 compared to the same period in 2019. The magnitude of this decrease is larger than during previous economic downturns or World War II. The timing of emissions decreases corresponds to lockdown measures in each country.''  However, we cannot assume that reductions in labor will generally have the same effect on carbon emissions as we observed in 2020.  This is because the pandemic lockdowns led to more than just reductions in labor: they also changed how people consumed, drove, and socially interacted with one another.  Thus a reduction in labor induced by a change in the tax code - which I model in this paper - may have an entirely different effect on emissions than one induced by a pandemic lockdown.

}

 %A large body of work exists that estimates the elasticity of the labor supply, the elasticity of driving with respect to the price of fuel and other costs, and the elasticity of driving with respect to income.  Furthermore, researchers have estimated the carbon emissions released by driving.  Together, this large empirical literature - some of which is reviewed below - can shed some light on how policies that change how much people work can also change how much pollution is released.  The contribution of this paper is to consider all these effects jointly, in the interest of shedding light on how changes in labor time affect pollution levels.  Furthermore, this paper's contribution is to develop a model so that we can consider how these different effects interact in equilibrium, rather than considering them in isolation from one another.  

 %This paper uses an OLG model calibrated so that it represents the United States in the year 2018.  People in the model can spend their time working, engaged in leisure, or driving.  Their money can be spent on driving (getting gas, buying and maintaining a car, etc.) or on general consumption; it can also be saved and earn interest.  I calculate the steady-state equilibrium of the model, and I then calculate the flow of carbon emissions when the model is in the steady-state.  I next perform ``experiments'' on the model by having the government implement policies (such as a tax on wages or mandatory-retirement age) that cause the agents to change how much they work; I then calculate the steady-states for these experimental parameterizations and the flow of emissions in these steady-states.  By comparing the carbon emissions in the experimental models to the original model, I can see if reductions in labor are associated with increases or decreases in carbon emissions.

I perform six experiments on the model.  The first and second experiment are to implement a wage tax of 25 percent.  In both cases, the government runs a balanced-budget.  In the first experiment, the tax revenue is all spent by the government on its own projects; in the second the tax revenue is equally distributed to all living agents.  Let's consider the second experiment first: in that experiment, the new policy (wage tax and then rebate) induces agents to work less and pollute less, thus suggesting that the reallocation-of-production effect did not counter the reduced-production effect.  This is, in fact, the result of all the experiments: whenever work is reduced by a significant amount, pollution is reduced as well, thus suggesting the reallocation-in-production effect is not particularly fearful.  

The first experiment - in which a wage tax is implemented and the government spends the revenue on government projects - is something of a bust because the policy change does not induce the agents to work less.  (As will be discussed below, this matches with some studies of labor market elasticity, especially those of prime-age males, which find that wage changes do not induce large movements in the number of hours worked.)  Interestingly, the tax nonetheless causes a small reduction in the level of pollution despite almost no effect on aggregate labor hours.  This is because the tax forces agents to spend less on consumption and driving and thus leads to a reduction in privately-caused pollution.  The new government spending leads to an increase in government-caused pollution; of course the overall result depends on our assumptions about how government spending causes pollution, which will be discussed below.

The third and fourth experiments implement an emissions tax.  That is, a consumption tax is added to general consumption and to driving, in proportion to the emissions caused by general consumption and to driving.  In the third experiment (like in the first experiment), the government spends the tax revenue on its own projects.  This experiment is similarly a bust in that the agents are not induced to work less in the aggregate.  But in the fourth experiment - in which the money is rebated equally to each living agent, as it was in the second experiment - the agents are induced to work less, and pollution drops as well.

In the fifth and sixth experiments, the government doesn't tax but instead directly puts restrictions on how much agents can work.  In the fifth experiment, a mandate is issued that agents must retire at a certain age.  In the sixth experiment, agents can work in every period but cannot work more than four-tenths of each period.  In both cases, the amount of labor drops significantly and so does pollution. 

Thus, the first and third experiments failed to induce the agents to work less.  But in all other experiments, the agents worked less and pollution dropped.  This demonstrates that in none of these cases studies does the reallocation-of-production mechanism reverse the effect of the reduced-production mechanism.  %This is consistent with empirical studies (discussed in Section \ref{sec:validation}) that show that people tend to drive more when they have more income; to my knowledge there is no evidence that driving goes up when people work less.

 In Section \ref{sec:model}, I will introduce this paper's model.  In Section \ref{sec:calibrate} I will calibrate the model.  In Section \ref{sec:validation} we will review how the model is validated by real-world data.  In Section \ref{sec:results} I will discuss the experiments and results.  In Section \ref{sec:conclusion} I will conclude.  %(TODO: make sure all these section numbers are right.)  (TODO: discuss education at some point?)  (TODO:  present stats on how much driving is commuting?  See email from Wednesday 2/23)


%\section{Literature Review}\label{sec:lit-review}
%
%
%
%The idea that a reduction in economic activity can lead to a reduction in pollution is well-illustrated by the lockdowns in the early part of the pandemic.  \citet{liu} found ``an abrupt 8.8\% decrease in global CO2 emissions (-1551 Mt CO2) in the first half of 2020 compared to the same period in 2019. The magnitude of this decrease is larger than during previous economic downturns or World War II. The timing of emissions decreases corresponds to lockdown measures in each country.'' 
%
%  Although the pandemic experience is interesting, we cannot assume that reductions in labor will generally have the same effect on carbon emissions as we observed in 2020.  This is because the pandemic lockdowns led to more than just reductions in labor: they also changed how people consumed, drove, and socially interacted with one another.  Thus a reduction in labor induced by a change in the tax code may have an an entirely different effect on emissions than one induced by a pandemic lockdown.
%
%
%\subsection{Literature that Estimates Elasticity}
%The experiments in this paper all involve changes in government policy, many of which change the prices of leisure and driving faced by the agents, as well as their income.  It is therefore essential that we consider how people respond to changes in these prices and in their income.
%
%Let's first consider the research on labor supply elasticity.  Most studies found that prime-age men had labor supply elasticities close to zero, implying that the only people who responded to wage changes were those on the fringes of the labor market, such as the second earner in two-earner households or older workers considering retirement.  For example, \citet{altug} and \citet{ziliak} are two studies that find prime-age married men provide labor with near-zero elasticity (Frisch elasticities of 0.14 and 0.16, respectively).\footnote
%{The Frisch elasticity measures how people change their labor supply in response to expected wage-changes.  Economic theory generally predicts that the labor supply will shift more dramatically in response to expected than unexpected wage-shifts, so an estimate of Frisch elasticity serves as a ceiling on how high Marshallian elasticity can be.} 
%According to \citet{keane}, these studies are flawed because they look only at the intensive margin (that is, they look at how the number of hours worked by an employed worker changes rather than seeing if wage-changes prompt people to switch between employment and non-employment). More recent papers that take into account the participatory margin find much larger Frisch elasticities.  \citet{erosa}, \citet{keane_wasi}, and \citet{iskhakov} find elasticies of 1.75, 0.74, and 1.5-1.8 among the male subgroups they study.\footnote
%{\citet{keane} also argues that many studies underestimate prime-age male labor supply elasticity because they only consider how work-hours change in response to wage-changes, as opposed to considering how men change their education and training levels in response to wage-changes.}
%
%\citet{wei} uses a dynamic model to estimate the price elasticity of gasoline, finding it is -0.2 in the short term and -0.5 in the long run.  \citet{bento} find elasticities between -0.25 and -0.30, and USDOE 1996 found -0.38.  \cite{berry} use Swedish household data and find the long-run fuel price elasticity of VMT (vehicle miles traveled) is -0.69 and the population's average income elasticity of VMT is 0.42.%\footnote{\url{https://docs.google.com/viewer?a=v&pid=sites&srcid=ZGVmYXVsdGRvbWFpbnxjaGFvd2VpZWNvbnxneDoyNGYwYmQ1ZmNhZDQzY2Rl} \url{https://papers.ssrn.com/sol3/papers.cfm?abstract_id=4154084}}
%Using odomoter readings of new vehicles registered in California and then later given a smog check, \citet{gillingham} finds a medium-run VMT fuel-price elasiticity for new cars of -0.22.%\footnote{\url{https://www.sciencedirect.com/science/article/abs/pii/S0166046213000653}}
%
%\subsection{Literature that Estimates The Contribution of Driving to Carbon Emissions}
%A goal of this paper is to examine how, when people work less, they choose to consume more-polluting things (such as flights in air planes) or less-polluting things (like meditation classes).  But to make this paper tangible, my model focuses specifically on how working less affects how much people drive private vehicles.  Driving private vehicles is a major source of carbon emissions, whether we look at international or national data.
%
%According to the IPCC's Sixth Assessment Report, in 2019 there were about 59 GtCO2eq anthropogenic (human-caused) carbon emissions.  That number includes emissions relating from fossil fuel combustion and industrial processes, as well as land use changes such as agriculture and forestry.  Of the 59 GtCO2eq, about ten percent originated from road transportation, making it a large source of emissions.  About 16 percent of emissions come from buildings, 22 percent from agriculture and land use, and 34 percent from industry, 5 percent from non-road transport, and 12 percent from other sources (page 2-31 of \citet{ipcc2}) Between 2010-2019, road transport was one of the fastest growing sources of emissions at 1.7 percent of the year, although this halted during the COVID-19 pandemic (page 10-10 of \citet{ipcc10}).
%% \footnote{page 10-10 of \url{www.ipcc.ch/report/ar6/wg3/downloads/report/IPCC_AR6_WGIII_Chapter_02.pdf}, page 2-31 and 2-19 of \url{www.ipcc.ch/report/ar6/wg3/downloads/report/IPCC_AR6_WGIII_Chapter_02.pdf}}
%\footnote{Note that in these numbers, energy and electricity are not counted as a separate sector because they are counted with the sector that uses them.  But however electricity and heat are counted, transport is about fifteen percent of total emissions.  See Figure 2.12 of \citet{ipcc2})}
%
%The United States EPA looks specifically at emissions from the United States in 2020 and says that transport makes up 27 percent of emissions, with over half coming from road transport.  25 percent is from electric power, 24 percent is from industry, 13 percent from commercial and residential, and 11 percent from agriculture.  Thus transport makes up a much larger share of US emissions than global emissions.  US emissions was growing from 1991 to 2004, in part because of the increasing popularity of fuel efficient vehicles.  While that trend has since waned, the rise of VMT has driven the trend of rising emissions from transportation.  "In terms of the overall trend, from 1990 to 2020, total transportation emissions have increased due, in large part, to increased demand for travel. The number of VMT by light-duty motor vehicles (passenger cars and light-duty trucks) increased by 30\% from 1990 to 2020, as a result of a confluence of factors including population growth, economic growth, urban sprawl, and periods of low fuel prices." \citep{epa_sources})  After 2004, US transportation emissions roughly stayed level, dipping during the Great Recession and then almost recovering afterward, until they steeply declined at the beginning of the pandemic.
%
%\subsection{Literature that Examines the Purpose Behind Driving}
%The US Department of Transportation, relying on data from the NHTS, publishes data on the purpose of trips in privately-owned vehicles.  If we exclude trips whose purpose is to go home, we find that 25.3 percent of trips are for work and 23 percent are for either meals or social/recreation.  The largest trip purpose is for shopping and errands at 29.7 percent.  \footnote{Other trip purposes include school or daycare (4.7), medical and dental (2.3), transporting someone else (12.9), and other (2.1)  See \citet{transportation}.}  We should be cautious in how we use these numbers to predict how a reduction in work hours would affect vehicle miles driven, in part because the data refers to to the purpose of trips rather than the purpose of miles.  Furthermore, it is difficult to ascertain the true ``cause'' of any particular trip because trips can be chained together (for example, a person using the car to run errands on the way home from work).  A reduction in work hours could change the number and times a person commutes to the workplace, which could cause the person to re-shuffle their use of the car in myriad ways. % \footnote{Much research on trip chaining exists}




 %A large body of work exists that estimates the elasticity of the labor supply, the elasticity of driving with respect to the price of driving (as represented, to use an imperfect example, by how the number of miles driven changes as the price of gasoline changes) and the elasticity of driving with respect to income.  Furthermore, researchers have estimated the carbon emissions released by driving.  Together, this large empirical literature can shed some light of how policies that change how much people work can also change how much pollution is released.  The contribution of this paper is to consider all these effects jointly, in the interest of shedding light on how changes in labor time affect pollution levels.  Furthermore, this paper's contribution is to develop a model so that we can consider how these different effects interact in equilibrium, rather than naively considering them in isolation to one another.  

%***
%To my knowledge, this paper is the first to develop a model which shows how a reduction in labor affects pollution levels.  \footnote{(But what about paper in lit review?)}  


%In this paper, I run ``experiments'' by having the government implement certain policies (such as tax changes) which induce agents to change how much they work; they may also change how much they drive, how much they consume, and how much they save.  These actions in turn affect how much pollution is released.  A large body of work exists that estimates the elasticity of the labor supply, the elasticity of driving with respect to the price of driving (as represented, to use an imperfect example, by how the number of miles driven changes as the price of gasoline changes) and the elasticity of driving with respect to income.  Furthermore, government agencies \footnote{Be more specific} publish estimates of the carbon emissions released by driving.  Together, this large empirical literature can shed some light of how policies that change how much people work can also change how much pollution is released.  The contribution of this paper is to consider all these effects jointly, in the interest of shedding light on how changes in labor time affect pollution levels.  Furthermore, this paper's contribution is to develop a model so that we can consider how these different effects interact in equilibrium, rather than naively considering them in isolation to one another.  

%Let's first consider the research on labor supply elasticity.  Most studies found that prime-age men had labor supply elasticities close to zero, implying that the only people who responded to wage changes were those on the fringes of the labor market, such as the second earner in two-earner households or older workers considering retirement.  For example, Altug Miller (1990) and Ziliak Kneisner (1999) are two studies that find prime-age married men provide labor with near-zero elasticity (Frisch elasticities of 0.14 and 0.16, respectively).\footnote
%{The Frisch elasticity measures how people change their labor supply in response to expected wage-changes.  Economic theory generally predicts that the labor supply will shift more dramatically in response to expected than unexpected wage-shifts, so an estimate of Frisch elasticity serves as a ceiling on how high Marshallian elasticity can be.} 
%According to Keane (2022), these studies are flawed because they look only at the intensive margin (that is, they look at how the number of hours worked by an employed worker changes rather than seeing if wage-changes prompt people to switch between employment and non-employment). More recent papers that take into account the participatory margin find much larger Frisch elasticities.  Erosa Fuster Kambourov (2016), Keane Wasi (2016), and Iskhakov Keane (2021) find elasticies of 1.75, 0.74, and 1.5-1.8 among the male subgroups they study.\footnote
%{Keane (2022) also argues that many studies underestimate prime-age male labor supply elasticity because they only consider how work-hours change in response to wage-changes, as opposed to considering how men change their education and training levels in response to wage-changes.}

%Wei (2013) uses a dynamic model to estimate the price elasticity of gasoline, finding it is -0.2 in the short term and -0.5 in the long run.  Bento and et al (2009) find elasticities between -0.25 and -0.30, and USDOE 1996 found -0.38.  Berry and Börjesson (2022) use Swedish household data and find the long-run fuel price elasticity of VMT (vehicle miles traveled) is -0.69 and the population's average income elasticity of VMT is 0.42.\footnote{\url{https://docs.google.com/viewer?a=v&pid=sites&srcid=ZGVmYXVsdGRvbWFpbnxjaGFvd2VpZWNvbnxneDoyNGYwYmQ1ZmNhZDQzY2Rl} \url{https://papers.ssrn.com/sol3/papers.cfm?abstract_id=4154084}}
%Using odomoter readings of new vehicles registered in California and then later given a smog check, Gillingham (2014) finds a medium-run VMT fuel-price elasiticity for new cars of -0.22.\footnote{\url{https://www.sciencedirect.com/science/article/abs/pii/S0166046213000653}}

%This paper, to my knowledge, is the first to use a calibrated model to show how a reduction in labor can affect the flow of carbon emissions.  It is close in spirit to Heijdra, Heijnen, and Kindermann (2014).  They develop a model in which agents cause pollution when they work and consume; furthermore, because the agents perceive leisure and a clean environment as complements, they tend to work more as the environment becomes more degraded, leading to a ``bad'' equilibrium in which the world is polluted and people have little leisure time.  The paper illustrates the idea that, through collective action, the agents can move to a preferred equilibrium with more leisure and less pollution.

%The idea that a reduction in economic activity can lead to a reduction in pollution is well-illustrated by the lockdowns in the early part of the pandemic.  Liu et al (2020) found ``an abrupt 8.8\% decrease in global CO2 emissions (-1551 Mt CO2) in the first half of 2020 compared to the same period in 2019. The magnitude of this decrease is larger than during previous economic downturns or World War II. The timing of emissions decreases corresponds to lockdown measures in each country.'' \footnote{\url{https://www.nature.com/articles/s41467-020-18922-7}}  Although the pandemic experience is interesting, we cannot assume that reductions in labor will generally have the same effect on carbon emissions as we observed in 2020.  This is because the pandemic lockdowns led to more than just reductions in labor: they also changed how people consumed, drove, and socially interacted with one another.  Thus a reduction in labor induced by a change in the tax code may have an an entirely different effect on emissions than one induced by a pandemic lockdown.

%As mentioned earlier, this paper focuses on private vehicles to represent high-polluting goods.  Much research shows that driving private vehicles is a major source of carbon emissions at both the international and national level.

%According to the IPCC (2021), in 2019 there were about 59 GtCO2eq anthropogenic (human-caused) carbon emissions.  That number includes emissions relating from fossil fuel combustion and industrial processes, as well as land use changes such as agriculture and forestry.  Of the 59 59 GtCO2eq, about ten percent originated from road transportation, making it a large source of emissions.  About 16 percent of emissions come from buildings, 22 percent from industry, and 34 percent from industry, 5 percent from non-road transport, and 12 percent from other sources.  Between 2010-2019, road transport was one of the fastest growing sources of emissions at 1.7 percent of the year, although this halted during the COVID-19 pandemic.  \footnote{page 10-10 of \url{www.ipcc.ch/report/ar6/wg3/downloads/report/IPCC_AR6_WGIII_Chapter_02.pdf}, page 2-31 and 2-19 of \url{www.ipcc.ch/report/ar6/wg3/downloads/report/IPCC_AR6_WGIII_Chapter_02.pdf}}
%\footnote{Note that in these numbers, energy and electricity are not counted as a separate sector because they are counted with the sector that uses them.  But however electricity and heat are counted, transport is about fifteen percent of total emissions.  (Figure 2.12 of IPCC(2021))}

%EPA looks specifically at emissions from the United States in 2020 and says that transport makes up 27 percent of emissions, with over half coming from road transport.  25 percent is from electric power, 24 percent is from industry, 13 percent from commercial and residential, and 11 percent from agriculture.  Thus transport makes up a much larger share of US emissions than global emissions.  US emissions was growing from 1991 to 2004, in part because of the increasing popularity of fuel efficient vehicles.  While that trend has since waned, the rise of VMT has driven the trend of rising emissions from transportation.  "In terms of the overall trend, from 1990 to 2020, total transportation emissions have increased due, in large part, to increased demand for travel. The number of vehicle miles traveled (VMT) by light-duty motor vehicles (passenger cars and light-duty trucks) increased by 30\% from 1990 to 2020, as a result of a confluence of factors including population growth, economic growth, urban sprawl, and periods of low fuel prices." (EPA)  After 2004, US transportation emissions roughly stayed level, dipping during the Great Recession and then almost recovering afterward, until they steeply declined at the beginning of the pandemic.

%The US Department of Transportation, relying on data from the NHTS, publishes data on the purpose of trips in privately-owned vehicles.  If we exclude trips whose purpose is to go home, we find that 25.3 percent of trips are for work and 23 percent are for either meals or social/recreation.  The largest trip purpose is for shopping and errands at 29.7 percent.  \footnote{Other trip purposes include school or daycare (4.7), medical.dental (2.3), transporting someone else (12.9) and other (2.1)}  We should be cautious in how we use these numbers to predict how a reduction in work hours would affect vehicle miles driven, in part because the data refers to to the purpose of trips rather than the purpose of miles.  Furthermore, it is difficult to ascertain the true ``cause'' of any particular trip because trips can be chained together (for example, a person using the car to run errands on the way home from work).  A reduction in work hours could change the number and times a person commutes to the workplace, which could cause the person to re-shuffle their use of the car in myriad ways.  \footnote{Much research on trip chaining exists}




%Labor supply- see notes I emailed to myself

%Driving elasticity -  see Wei (2013) and Song and Wei (2018) already in lit review, and also Wei's website in general.  Beside looking at estimates of elasticity, we can also look at estimates of how much driving is done to get to work (commuting) and how much is done on the job.

%Pollution, see EPA website and also \url{https://www.ipcc.ch/report/ar6/wg3/downloads/report/IPCC_AR6_WGIII_Chapter_10.pdf}
%for international, see Page 10-10 of this: \url{https://www.ipcc.ch/report/ar6/wg3/downloads/report/IPCC_AR6_WGIII_Chapter_10.pdf}

%OPTIONAL: look at how people spend time when leisure increases, if such papers exist?

%Some papers have looked at how work contributes to pollution. Heijdra, Heijnen, and Kindermann (2014) develop a model in which agents cause pollution when they work and consume; furthermore, because the agents perceive leisure and a clean environment as complements, they tend to work more as the environment becomes more degraded, leading to a ``bad'' equilibrium in which the world is polluted and people have little leisure time.  The paper illustrates the idea that, through collective action, the agents can move to a preferred equilibrium with more leisure and less pollution.

%Many papers have looked at the effect of the COVID-19 pandemic on the natural environment.  For example, XXX finds YYY.  (cite).  These papers show that when people work less and drive less, less pollution is possible.  However, the pandemic is a unique situation; when areas were formally or informally ``locked down'', people were restricted (or felt restricted) in what places they could drive to and what goods and services they could consume.  Thus, we cannot make general characterizations of how working less affects driving or affects pollution, based on the pandemic.



%***
%Not surprisingly, I find that large reductions in labor are associated with drops in pollution.  This is easy to understand; if people did not work at all, then nothing would be produced and no emissions would be released at all (at least, none from production).  However, for small reductions in labor, the second mechanism can sometimes dominate the first mechanism - leading to an overall increase in pollution - sometimes not.



%\section{Literature Review} \label{sec:lit-review}
%Although the tradeoff between leisure and consumption is commonly taught in basic economics courses, and the tradeoff between consumption and a clean environment is widely discussed in the popular press; it is somewhat rarer for the connection between leisure time and a clean environment to be explicitly studied and discussed.  Heijdra, Heijnen, and Kindermann (2014) are an exception; they produce a modified Ramsey-Cass-Koopmans model in which leisure and environmental quality (non-pollution) are complements.  Their model has multiple equilibria: a ``bad'' equilibrium in which agents experience little leisure and live in a polluted world, and a ``good'' equilibrium which is a ``utopia'' of leisure and a clean environment.  To escape the bad equilibrium, the social planner must coordinate pollution abatement and pigovian taxes.  (Individuals do not have an incentive to minimize pollution on their own.). But as society moves away from the bad equilibrium, the optimizing social planner can stop pollution abatement and reduce pigovian taxes; this is because self-interested individuals reduce their amount of work and pollution as the environment gets cleaner.\\
%\\
%In my view, Heijdra, Heijnen, and Kindermann (2014) are not convincing that leisure and the environment are complements.  In the model developed in this (my) paper, people prefer to live in a clean environment, but their behavior is not at all affected by the current state of the environment.  This is because no individual can meaningfully affect the pollution level, and how they rank the decisions which they do have control over is not affected by the state of the environment.\\
%\\
%Wei (2013) studies the effects of the costs of driving on how much people drive, how much gasoline they use, and the fuel-efficiency of their vehicles; they are particularly interested in the consequences of gasoline taxes and higher CAFE standards.  In their model, driving affects utility in two ways: people get positive utility from driving more $miles$, but they lose utility from spending $time$ driving.  Vehicle choices and vehicle lifetimes are explicitly modeled; however their model does not focus on the choice of transportation mode (driving vs public transportation), does not have people explicitly retire (although they can reduce their labor hours), and does not suggest that people's preferences change depending on if they are working or retired.\\
%\\
%Song and Wei (2018) use time use surveys to study changes in the amount of time Americans have spent traveling; they look at total traveling time and do not look at mode choice (driving, biking, bus, etc.). They find that travel time increased from 1975 to 1993 due to demographic changes, but then declined from 1993 to 2013.  The decline was not due to demographic changes because it affected all demographics roughly the same; it also was not because Americans economized on travel for particular activities.  The decline is explained by Americans shifting the amount that they do certain activities; they shifted more toward activities which do not require a lot of travel (such as video games, home entertainment systems, and other home leisure activities).\\
%\\
%Borck (2019) is interested in the effects of public transportation policies on urban pollution.  The author is skeptical of previous studies which have found that better funding for public transportation can lead to less pollution as these studies did not establish general equilibrium models.  In their paper, they find that increased funding for public transportation has a minimal impact.  However, this is because their model assumes that the funding is from an outside source; the agents in the model therefore have newfound wealth with which to buy other pollution-causing goods, such as larger houses.  \\
%
%%Somewhat bizarrely, the author acknowledges the problem, and then runs a bunch of extensions to the baseline model which never correct the problem.\\
%%TO DO: make sure you understand this paper
%
%My complaints about Borck (2019) aside, the lesson here is that reduction of production-caused pollution requires people to spend less time working, or to spend their working time producing goods that are less-polluting.  An outside gift of less-polluting goods (such as increased funding for public transportation) does not reduce pollution if people use their resources to pollute in other ways.\\
%\\
%Yang, Wang, Liu, and Zhou (2018) develop a model which focuses on the factors which influence how people decide what mode of transportation to take: the characteristics of the trip itself (work or leisure), the person's demographics, the characteristics of the mode (speed, cleanliness, cost), and the person's attitude toward different modes and toward environmental causes.  After surveying people in Beijing malls about where and how they traveled the previous day (and why), they estimate how different factors affect mode choice and then predict how improvements to Beijing's public transportation would affect mode choice and pollution in Beijing.  \\
%\\
%This (my) paper's baseline model does not focus on trip characteristics as explicitly as Yang, Wang, Liu, and Zhou (2018) do, but it does allow for working and retired people to have different preferences regarding different travel modes. 



\section{Model}\label{sec:model}

\subsection{Key Features of the Model}

This paper uses an OLG (Overlapping Generations Model).  Some key features of the model are shown in Figure \ref{fig:time_diagram}.   

\begin{figure}[H]
\hspace{0.5in}
	\includegraphics[height=6.4in,width=9in]{./time_diagram.png}%{./tax_func_holter.pdf}
\caption{\scriptsize  Agents are originally endowed with time (upper left).  The time can be used for labor, leisure, or driving.  Labor produces a money wage.  Money can be used for driving (which costs both time and money), or on consumption.  Both driving and consumption cause different amounts of pollution; leisure does not cause pollution.  The three items in shaded boxes (leisure, driving, and consumption) are sources of utility.
}\label{fig:time_diagram}
\end{figure}


Starting in the upper-left, we see that agents are originally endowed with time, but not money.  The time can be spent three ways: on labor (which produces money), driving, or on leisure.  Money can be spent on either the general consumption good (which represents all goods and services except driving private vehicles), or on driving.  Note that driving costs both time and money.  Consumption only costs money (however, that money must be acquired through labor-time, so indirectly consumption costs time.)

The three-shaded boxes represent the three things which give the agent utility: leisure, driving, and general consumption.  Both driving and consumption cause pollution.  Thus, the least-polluting way for an agent to spend time is to spend it in leisure.  Time spent driving directly causes pollution; time spent in labor earns money which is eventually spent on driving or consumption, both of which cause pollution.  (The amounts of pollution caused are dependent on the calibrations of $e_c$ and $e_m$, which are discussed below.  Since driving is calibrated as the higher-polluting good, we know that agents can reduce their emissions by increasing their leisure time, and by spending their money on less-polluting consumption, rather than driving.)
	
What follows is a specific description of the model.

\subsection{Time}

Time is discrete, has an infinite horizon, and is indexed by $t$.  The first period is $t=0$.

\subsection{Agent lifecycle}

Each period, a unit continuum of agents is born with age 0.  Each agent lives for $N$ years.  Thus, at the end of each period, all agents of age $N-1$ die.  Furthermore, during any period $t$ there are $N$ generations alive.

\subsection{Agent age and birth year}

I use $b_i$ (or $b$ when the subscript is not needed for clarity) to mean the period in which agent $i$ is born.  Thus in period $t$, an agent $i$'s age is $a_i = t - b_i$.

\subsection{Agent's time endowment}

Each period, each agent is endowed with 1 unit of time that can be spent on leisure, work, car transportation, or any combination.

%\subsection{Time and Lifespan}

%Time is discrete, and has an infinite horizon.  The first period is $t=0$.

%During any period $T$, there are $N$ generations alive.  Each period, a unit continuum of agents is born with age 0.  Each agent lives for N years.  Thus, at the end of each period, all agents of age $N-1$ die.

%Each period, each agent is endowed with 1 unit of time that can be spent on leisure, work, car transportation, or any combination.

%TODO: Dr. Silos commented that I should explain if retirement is exogenous or endogenous

\subsection{Productivity}

When an agent $i$ is born, the agent's productivity $w_i$ is drawn from the exogenous distribution $f(w)$ and never changes during the agent's life.  Productivity is different across individual agents, but each generation has the same distribution.

\subsection{Market wealth}

Let $D_a$ indicate an agent's wealth at the beginning of the period in which they are age $a$.  Each agent is born with zero wealth and is required to have nonnegative wealth at death.  Thus,
\begin{equation}D_0=0\end{equation}
\begin{equation}D_N \ge 0\end{equation}
At the end of each period, an agent's wealth is multiplied by $1+r$, where $r$ is the exogenous interest rate.

\subsection{Utility}

All agents have identical utility functions.  During any period $t$, each agent derives utility from consuming consumption good $c$, leisure time $L$, and driving services $m$.  The agent's utility is increasing in all three arguments; marginal utility is decreasing in all three arguments.

Leisure is defined as time in which the agent is not working and not driving.  Driving, $m$, is measured in distance traveled.  This aspect of the model, inspired by \citet{wei}, is meant to convey the idea that agents derive utility from the \emph{distance} they travel but experience the \emph{time} spent traveling as an opportunity cost (foregone leisure).

The idea is not that people literally get more pleasure from traveling more distance; the idea is that when people travel more distance they are able to travel to more places.  

%The different weights for $m^d$ and $m^p$ capture the idea that some places can be accessed better by one mode of transportation than the other.  It also captures that one mode of transportation may take more circuitous routes and thus a mile traveled in one mode is not equal to a mile traveled in the other.  Finally, agents may weight the different modes of transportation differently because of other aspects (such as risk of an accident, cleanliness, pleasantness of social interaction, etc.)

\subsection{Prices}

Each argument in the utility function has a market price and a time-cost.  To be overly simplistic, this captures the idea that each argument costs some money and some time.  However, I use the term `market price', rather than `money price' because this model abstracts away from money; there is no currency or monetary systsem in this model.  The market price is how much market wealth, measured in units of current consumption $c_t$, must be spent on the activity.  The time-cost is how much of one's time endowment must be spent on the activity (recall that each period each agent has one unit of time endowment.)  
%\begin{table}[H]
%\begin{tabular}{l l l l}
%\toprule
%\textbf{Variable} & \textbf{Description} & \textbf{Price} & \textbf{Time-Cost}\\
%\midrule
%$c$ &  Consumption & 1 & 0\\
%$L$ & Leisure & 0 & 1\\
%$m$ & Driving Distance & $q$ & $h$\\
%%$m^p$ & Public Trans Distance & $q^p$ & $h^p$\\
%\bottomrule
%\end{tabular}
%\caption{Prices}
%\end{table}


\noindent
%\textcolor{red}{\Large\textbf{after}}
\begin{table}[!h]

\centering
\begin{threeparttable}


\begin{tabular}{cccc}
\toprule
{Variable} & {Description} & {Price} & {Time-Cost}\\
\midrule
$c$ &  Consumption & 1 & 0\\
\addlinespace
$L$ & Leisure & 0 & 1\\
\addlinespace
$m$ & Driving Distance & $q$ & $h$\\
%$m^p$ & Public Trans Distance & $q^p$ & $h^p$\\
\bottomrule
\end{tabular}
\caption{Prices}
\label{table:prices}
\end{threeparttable}
\end{table}

\subsection{Utility Function}

\subsubsection{General description of utility function}

An agent's lifetime utility function, $U$, is dependent on general consumption, leisure, and driving distance in each period of the agent's life.  The lifetime utility function $U$ is equal to a discounted stream of period utility functions $u$.  Total consumption in a particular period - which includes general consumption, leisure, and driving distance - is represented by $C$, which takes the form of a nested Constant Elasticity of Substitution (CES) function.

Why use a nested CES function?  We are interested in the elasticity of substitution between general consumption, leisure, and driving.  An ordinary (non-nested) CES function, which took these three items as arguments, would be overly restrictive.  Specifically, such a model would imply that there is one elasticity of substitution between all three arguments, and thus would eliminate crucial nuance.  

Because the elasticity of substitution between driving ($m$) and leisure ($L$) is of high interest, they are nested together in a CES function, $g(L, m)$, which is the ``inner nest'' of the nested CES function.  The $g$ function is then paired with general consumption, $c$, in the index of total consumption, $C$, which is the ``outer nest''.  Thus $c$ and $g(L, m)$ appear in the $C = C(c, g(L, m))$ function, which appears in the period utility function $u=u(C)$, which itself appears in lifetime utility $U$.
% $\delta$ is the elasticity of substitution between $c$ and $g()$, and $\rho$ is the elasticity of substitution between $L$ and $m$.

\subsubsection{Brief note on notation}

I use a bold variable to represent a vector of values across the $N$ periods of an agent's life.  For example, a bold letter $\boldsymbol{c}$ stands for a vector showing general consumption in each period of an agent's life.

\begin{equation}\boldsymbol{c} = c_1, ..., c_N\end{equation}

\subsubsection{Specific description of utility function}
An agent's lifetime utility function is:

\begin{equation}
U = U\Big(\boldsymbol{c}, \boldsymbol{L}, \boldsymbol{m}\Big) =\sum_{a=0}^{{N-1}} \beta^{a-1} u\Big(c_{a}, L_{a}, m_{a}\Big)\label{eq:U}
\end{equation}
where $\beta$ is the exogenous discount factor, $u()$ is the period utility function, and $a$ is a subscript indicating the agent's age.  

The agent's period utility function is:
\begin{equation}
u(c_{a}, L_{a}, m_{a})=
   \begin{cases} 
      \frac{C(c_{a}, L_{a}, m_{a})^{1-\sigma}}{1-\sigma} & \text{for } \sigma \ne 1 \\
      \\
     \log{C(c_{a}, L_{a}, m_{a})} & \text{for }\sigma = 1\\ 
   \end{cases}
\end{equation}
where $\sigma \in [0, \infty]$ is the exogenous parameter of intertemporal substitution, with a low value indicating high intertemporal substitution.
$C()$ is a consumption index for consumption of goods, leisure, and driving, and takes the form of the CES (Constant Elasticity of Substitution) utility function.
\begin{equation}
%\[
C\Big(c_{a}, L_{a}, m_{a}\Big)=
   \begin{cases} 
       \Bigg(\kappa_c (c_{a})^\frac{\delta-1}{\delta} + \kappa_g g\Big(L_{a}, m_{a}\Big)^\frac{\delta-1}{\delta} \Bigg)^\frac{\delta}{\delta-1}& \text{for } \delta \ne 1 \\
      \\
     (c_{a})^{\kappa_c}\Big(g(L_{a}, m_{a})\Big)^{\kappa_g} & \text{for } \delta = 1\\ 
   \end{cases}
%\]
\label{eq:C}
\end{equation}

where $\delta \in [0, \infty]$ is the exogenous elasticity of substitution between $c$ and $g()$, and $\kappa_c, \kappa_g  \in (0, 1)$ are exogenous share parameters, with  $\kappa_c +\kappa_g = 1$.  $g()$ is a consumption index for distance traveled via driving, and leisure. 

\begin{equation}
%\[
g\Big(L_{a}, m_{a}\Big)=
   \begin{cases} 
       \Bigg(\kappa_L (L_{a})^\frac{\rho-1}{\rho} + \kappa_m (m_{a})^\frac{\rho-1}{\rho}\Bigg)^\frac{\rho}{\rho-1}& \text{for } \rho \ne 1 \\
      \\
     (L_{a})^{\kappa_L} (m_{a})^{\kappa_m} & \text{for } \rho = 1\\ 
   \end{cases} 
%\]
\label{eq:g}
\end{equation}

where $\rho \in [0, \infty]$ is the exogenous elasticity of substitution between $L$ and $m$ and $\kappa_L, \kappa_m  \in (0, 1)$ are exogenous share parameters for leisure and driving, with $\kappa_L+\kappa_m=1$.  

%\subsection{Nested Utility Function}
%Explain nested utility function here!

\subsection{Agent's Problem}

Below I describe the agent's problem.  For clarity I give each choice-variable two subscripts: one for the agent's age and one for the time period.  For an agent $i$ born in period $b_i$, the time period at age $a$ will always be $b_i+a$.  For example, an agent's leisure at age $a$ is $L_{a, b+a}$ and their leisure in all periods of life is: $\boldsymbol{L}=L_{0, b},...,L_{N-1, b+N-1}$.

Each agent $i$, born in a particular period $b_i$ and assigned an unchanging wage $w_i$, solves the following problem.  Note that for readibility I have removed the subscript $i$.  

\begin{equation}\max_{\boldsymbol{c}, \boldsymbol{L}, \boldsymbol{m}, \boldsymbol{l}} U\end{equation}

subject to the following constraints:

%An agent $i$, who is assigned an unchanging $w_i$ at birth, solves the following problem:
%$$\max_{x_{1},...,x_{N}} U$$
%subject to
\begin{enumerate}
\item \textbf{Budget constraint:}

%$$\sum_{a=1}^N \frac{wl_a(1-\tau_w)+T_{t(a)}}{(1+r)^{(a-1)}}  \ge \sum_{a=1}^N \frac{c_a(1+\tau_e e_c)+m_a(q+\tau_e e_m)}{(1+r)^{(a-1)}}$$

\begin{equation}\sum_{a=0}^{N-1} \frac{wl_{a, b+a}}{(1+r)^{a}}  \ge \sum_{a=0}^{N-1} \frac{c_{a, b+a}+m_{a, b+a}q}{(1+r)^{a}}\label{eq:budget_constraint}\end{equation}

The left side of the equation shows the agent's lifetime income.  An agent with wage $w$ who works $l_{a, b+a}$ time receives  $wl_{a,b+a}$ wages, at age $a$.

The right side show's the agent's lifetime spending.  General consumption is defined as having a price of 1, and the price of driving is exogenous $q$.

The exogenous interest rate is $r$.


%where $\tau_w$ is an exogenous wage tax, $\tau_e$ is an exogenous carbon tax, $e_c$ is the exogenous emissions per unit of consumption, and $e_m$ is the exogenous emissions per mile driven,  and $r$ is the exogenous interest rate, and R is the rebate amount


\item \textbf{Time constraint:}
At each age $a$, an agent can divide 1 unit of time between leisure, labor, and driving.

\begin{equation}1 = L_{a, b+a} + l_{a, b+a} + m_{a, b+a}h_{b+a}\text{ for each } a\end{equation}

Note that the time it takes to drive one mile is endogenous.  Although it can vary from period to period, all agents have the same time-cost of driving one mile during a given period.  At time $t$, it takes each agent $h_t$ to drive one mile.  Thus, at age $a$, it takes an agent $h_{b+a}$ to drive one mile.
\end{enumerate}

\subsubsection {How Government Policies Modify the Agent's Problem}
\underline {Wage Taxes}

When the government implements a wage tax, the budget constraint (Equation \ref{eq:budget_constraint}) becomes:

\begin{equation}\sum_{a=0}^{N-1} \frac{wl_{a, b+a} (1-\tau_w)+T_{b+a}}{(1+r)^{a}}  \ge \sum_{a=0}^{N-1} \frac{c_{a, b+a}+m_{a, b+a}q}{(1+r)^{a}}\end{equation}

The exogenous wage tax is $\tau_w$; thus an agent with wage $w$ who works $l_{a, b+a}$ time receives  $wl_{a,b+a}(1-\tau_w)$ amount of take-home wages, at age $a$.  

The government's policy may involve returning the collected taxes to the people.  If so, during each period $t$, every living agent receives a transfer payment $T_t$, which is determined endogenously. The transfer payment can vary from period to period, but during a given period it is the same for every living agent.  This means that an agent receives a transfer payment of $T_{b+a}$ at age $a$.

Alternately, the government may not return the collected taxes; the government may choose instead to spend the money on its own projects, which will be described further below.  If so, $T_t$ = 0 for all $t$.

\vspace{5mm}
\underline {Carbon Taxes}

When the government implements a carbon tax, the budget constraint (Equation \ref{eq:budget_constraint}) becomes:

\begin{equation}\sum_{a=0}^{N-1} \frac{wl_{a, b+a}+T_{b+a}}{(1+r)^{a}}  \ge \sum_{a=0}^{N-1} \frac{c_{a, b+a}(1+\tau_e e_c)+m_{a, b+a}(q+\tau_e e_m)}{(1+r)^{a}}\end{equation}

$\tau_e$ is an exogenous carbon tax, $e_c$ is the exogenous emissions per unit of consumption, and $e_m$ is the exogenous emissions per mile driven.  General consumption is defined as having a price of 1 plus the carbon tax, and the price of driving is exogenous $q$ plus the carbon tax.

Just as with the wage tax, the government may choose to return the collected taxes to the people, in which case every living agent receives a transfer payment $T_t$.  If the government decides to spend the money on its own projects, then $T_t=0$ for all $t$.

\vspace{5mm}
\underline{Retirement Constraint}

The government can impose a mandatory retirement age, exogenous $y \in \{1,...,N\}$.  If so, the following constraint must be respected (in addition to the time and budget constraints):
\begin{equation} l_{a, b+a} = 0 \text{ for each } a \in \{y,...,N\}\end{equation}

\vspace{5mm}
\underline{Work-maximum constraint}

The government may choose to impose a maximum amount that people can work, $Y$.  If so, the following constraint must be respected:

\begin{equation} l_{a, b+a} \le Y \text{for each } a\end{equation}






\subsection{Utility Maximization}

Agents always know the value of the transfer payment $T_t$ and the time-cost of driving, $h_t$, for all present and future $t$.  When solving their optimization problem, agents take the values of these endogenous variables as given, as well as their own wage $w_i$ (which is assigned at birth and unchanging), and all exogenous variables. 

For an agent $i$ born in period $b_i$ and assigned wage $w_i$, let $\boldsymbol{h}=h_b,...,h_{N-1}$ and  $\boldsymbol{T}=T_b,...,h_{T-1}$ be the time-costs of driving and the transfer payment during each period of their life, respectively.  We can then write this agent's optimal choices as the following functions (omitting the $i$ subscript for clarity):

\begin{equation}\boldsymbol{c}^* = \boldsymbol{c(}w, \boldsymbol{h}, \boldsymbol{T}\boldsymbol{)}\end{equation}
\begin{equation}\boldsymbol{L}^* = \boldsymbol{L(}w, \boldsymbol{h}, \boldsymbol{T}\boldsymbol{)}\end{equation}
\begin{equation}\boldsymbol{m}^* = \boldsymbol{m(}w, \boldsymbol{h}, \boldsymbol{T}\boldsymbol{)}\end{equation}
\begin{equation}\boldsymbol{l}^* = \boldsymbol{l(}w, \boldsymbol{h}, \boldsymbol{T}\boldsymbol{)}\end{equation}

Since all agents born in a certain period face the same values of $\boldsymbol{h}$ and $\boldsymbol{T}$, we can also say that the optional choices depend on a the period in which an agent was born, $b$.  An agent's optimal choices for a specific time period depend on the agent's wage, age during that period, and in what period they were born.

\begin{equation}c_{a, b+a}^* = c(w, a, b)\end{equation}
\begin{equation}L_{a, b+a}^* = L(w, a, b)\end{equation}
\begin{equation}m_{a, b+a}^* = m(w, a, b)\end{equation}
\begin{equation}l_{a, b+a}^* = l(w, a, b)\end{equation}

\subsection{Measuring Certain Aggregates (in equilibrium)}

In order to proceed, we must define certain aggregates.  In equilibrium, an agent's decisions depends solely on their wage, age, and period of birth.  We can therefore use $x^*_{w, a, b}$ to represent the optimal choice of an agent with wage $w$ and age $a$ and born in period $b$.  At any particular time $t$, $ \frac{1}{N}$ of the living population is age $a$ and was born in period $t-a$.  Furthermore, $f(w)$ of the population has wage $w$.  
\begin{enumerate}
\item At time $t$, the aggregate number of miles driven are:
\begin{equation}
A^m_t = \sum_{a=0}^{N-1} \Bigg( \int\limits_w m_{w,a,t-a}^* f(w) \text{ } dw \Bigg) \frac{1}{N}\label{eq:Ad}
\end{equation}
\item At time $t$, the aggregate amount of general consumption is:
\begin{equation}
A^c_t = \sum_{a=0}^{N-1} \Bigg( \int\limits_w c_{w,a,t-a}^* f(w) \text{ } dw \Bigg) \frac{1}{N}\label{eq:Ac}
\end{equation}
\item At time $t$, the aggregate amount of wages is:
\begin{equation}
A^w_t = \sum_{a=0}^{N-1} \Bigg( \int\limits_w l_{w,a,t-a}^*w f(w) \text{ } dw \Bigg) \frac{1}{N}\label{eq:Aw}
\end{equation}
\item At time $t$, the aggregate amount of collected taxes is the sum of wage taxes, emission taxes from general consumption, and emission taxes from driving.
\begin{equation}
A^\tau_t = A^l_t  w \tau_w + A^c_t  e_c \tau_e  + A^m_t  e_m \tau_e\label{eq:As}
\end{equation}

where $\tau_w, \tau_e$ are the exogenous wage and emissions tax, respectively.  $e_c$ and $e_m$ are the exogenous amount of emissions per unit of general consumption and per mile driven, respectively.
\end{enumerate}%do the 1/N belong?  Or does each generation count as 1?  TODO

\subsection{Government Spending and Transfers}

Our model must specify what is done with the tax money collected.  We will explore two possibilities: that the government spends all the tax money collected on its own projects, or that all tax money is transferred to the living population, with each agent getting an equal transfer.  Either way, it is assumed that the government always runs a balanced budget.

\begin{enumerate}
\item If the government spends all tax money on its own projects, then government-spending is equal to tax money collected, and transfers are equal to zero.  For all $t$,
\begin{equation}G_t=A^\tau_t\end{equation}
\begin{equation}T_t=0\end{equation}
\item  If the government transfers all tax money equally to each agent, then for all $t$:
\begin{equation}G_t=0\end{equation}
\begin{equation}
T_t=A^\tau_t\label{eq:Tt}%TODO check this is correct
\end{equation}
\end{enumerate}

\subsection{Congestion Externality: the time-price of driving}
The time it takes to drive a mile, during a given period $t$, is a decreasing function of the total amount of miles driven during that period.

\begin{equation}
h_t = \frac{A^m_t}{\omega}\label{eq:hd}
\end{equation}

where exogenous $\omega$ represents road infrastructure quality.

Note that each agent has an infinitesimal effect on the time-price of driving $h_t$ and the transfer payment $T_t$.  Agents always take those values as given when solving their optimization problem.

\subsection{Carbon Emissions Per Time Period}

$e_c$, $e_m$, and $e_G$ are exogenous variables that represent the amount of emissions per unit of general consumption, per mile driven, and per unit of government spending, respectively.  In a given time period $t$, the total flow of carbon emissions is:

\begin{equation}E_t = A^c_t e_c + A^m_t e_m + G e_G\end{equation}

\subsection{Steady-State Equilibrium (SSE)}

Intuitively, a steady-state equilibrium, or SSE, occurs when (1) each agent is making optimal choices, and (2) no aggregate variable changes from period to period.  In an SSE, an individual agent may choose to work or consume different amounts across different periods of life, but in the aggregate variables do not change.  Thus, the model-economy is in an SSE if the agents are making optimal choices consistent with unchanging aggregate values, and the unchanging aggregate values are consistent with the optimizing choices made by the agents.  I now give a more specific definition.


I will first define a steady-state equilibrium candidate; I will then define a steady-state equilbrium.  An SSE candidate consists of:

\begin{enumerate}
\item A value of $\boldsymbol{c}$ for every $w$ with support in $f(w)$
\item A value of $\boldsymbol{L}$ for every $w$ with support in $f(w)$
\item A value of $\boldsymbol{m}$ for every $w$ with support in $f(w)$
\item A value of $\boldsymbol{l}$ for every $w$ with support in $f(w)$
\item a single value of $h$
\item a single value of $T$
\end{enumerate}

An SSE-candidate is an SSE if and only if:
\begin{enumerate}
\item  Suppose $h_t=h$ for all $t$, and $T_t=T$ for all $t$.  The SSE-candidate maps each value of $w$ to the optimal choices of $\boldsymbol{c}$, $\boldsymbol{L}$, $\boldsymbol{m}$, and $\boldsymbol{l}$ for an agent born with wage $w$, given those values of $h$ and $T$.
\item   The values of $\boldsymbol{m}$ must imply that $h_t=h$ for all $t$, using Equations \ref{eq:Ad},and \ref{eq:hd}.
\item  The values of $\boldsymbol{c}$, $\boldsymbol{m}$, and $\boldsymbol{l}$ must imply that $T_t=T$ for all $t$, using Equations \ref{eq:As} and \ref{eq:Tt}.  (Note that this criteria is only relevant if the government's policy is to transfer collected tax money directly to the agents.  Otherwise $T_t=0$ for all $t$ and this criteria can be ignored.)
\end{enumerate}


%Let $x_{wa}$ be a quadruple $(c_{wa}, L_{wa}, m_{wa}, l_{wa})$
%An SSE-candidate is a set of:\\
%\begin{enumerate}
%\item $x_{wa}$ for all $w$ with support in $f(w)$, and all ages $a$ in $\{1,..., N\}$.\\
%\item $h$.
%\end{enumerate}
%
%An equilibrium-candidate is an SSE iff:
% \begin{enumerate}
%\item Given the value of $m_{wa}$ in the set of $x_{wa}$, the values of $h$ must be equal to the value calculated using Equations \ref{eq:Ad},and \ref{eq:hd}.
%\item Given $h$, $x_{wa}$ are the optimal solutions to the agent's problem.
%\end{enumerate}
%
%\subsection{Carbon Emissions Per Time Period in SSE}
%
%I've already given the formula for the aggregate distance traveled by driving (\ref{eq:Ad}).  Similarly, the aggregate amount of non-transportation consumption is:
%$$A^c = \sum_{a=1}^N \Bigg( \int\limits_w c_{wa} f(w) \text{ } dw \Bigg) \frac{1}{N}\label{eq:Ac}$$\\
%where $c_{wa}$ is the non-transportation consumption by agents with productivity $w$ and age $a$
%The aggregate amount of taxes collected is:
%$$A^\tau = \sum_{a=1}^N \Bigg( \int\limits_w (l_{wa}w\tau_w + c_{wa}e_c\tau_e + m_{wa}e_m\tau_e) f(w) \text{ } dw \Bigg) \frac{1}{N}\label{eq:As}$$\\
%
%In other words, it is the sum of the wage tax collected, the carbon tax collected on consumption, and the carbon tax collected for miles driven.
%
%The model assumes the government runs a balanced budget, so government spending $G=A^s$.  The government spending causes pollution but otherwise ``goes into a hole'' and doesn't affect the agents in any way.
%
%$e_c$, $e_m$,  and $e_G$ are exogenous parameters that indicate the amount of carbon emissions released per unit of non-transportation consumption, per mile driven, and per unit of government spending, respectively.  The total amount of carbon emissions released per time period is equal to:
%$$A^c e_c + A^m e_m + G e_G$$
%Note that this is meant to represent carbon emissions released during production, consumption, and disposal.  For example $e_m$ is meant to represent the carbon emissions per mile driven in car, and thus includes emissions that occur as material is mined, the car is manufactured, and the car is driven.





\section{Calibration}\label{sec:calibrate}

I will now discuss how each exogenous variable is calibrated.  Note that each time period is set equal to four years.\footnote{This shortens the computing time.}  Furthermore, each day is assumed to have only 12 hours available for work, leisure, and driving; the remaining twelve are assumed to be spent on sleep or personal activity.  Since each period is 4 years, this implies that there are about 17,532 hours per period.
\begin{itemize}
\item$\boldsymbol{N}$ is calibrated to \textbf{16}.  According to \citet{ssa}, the average 16-year-old has approximately 64 years of life remaining.\footnote{For males it is 60.90; for females it is 65.86 years.}  Those 64 years are equal to $64/4=16$ time periods.  \\  
\item $\boldsymbol{r}$ is calibrated to \textbf{21.55 percent}.  The Wall Street Journal's prime rate, which is meant to represent the rate at which banks lend to their most-favored customers, from June 14, 2018 to September 27, 2018 was 5 percent.  Since each time period represents 4 years, $r = 1.05^4-1 = 0.2155$.  During 2018, the rate ranged from 4.75 to 5.5 percent.  \citep{prime} \\
\item $\boldsymbol{\beta}$ is calibrated to \textbf{0.85}.  0.96 is the standard discount factor for 1 year in the literature.  Since each time period represents 4 years, we must use the four-year discount factor: $0.96^4=0.85$.\\ %TODO
\item $\boldsymbol{\sigma}$ is calibrated to $\boldsymbol{2}$, which is standard in the literature.\footnote{This is discussed on page 221 of \citet{nicholson}.  "Empirical evidence is generally consistent with values of $R$ in the range of -3 to -1."  The $\sigma$ used in this paper, when translated into the language of \citet{nicholson},  would be equal to $1-R$, which means between 2 and 4.}%TODO
%\item The wage distribution, $\boldsymbol{f(w)}$, is calibrated to the log-normal distribution $\boldsymbol{\log(w)\sim \mathcal{N}(\log(426,612),\,\sigma^{2})}$.  According to the 2018 ATUS, the mean wage was 24.35.  Multiplying this by 12 because there are 12 available hours in the day, times 365 times 4 because each period represents 4 years with 365 days each, we get that the mean wage is 24.35*12*365*4=426,612.  %TODO: explain how variance is calculated and check this whole section.
\item The wage distribution, $\boldsymbol{f(w)}$, is calibrated to the log-normal distribution $\boldsymbol{\log(w)\sim \mathcal{N}(\log(428,131), 63,840,961,685)}$.  The mean hourly wage is \$24.42, and 428,131 was calculated by multiplying the hourly wage 24.42 times 12 hours per day 365.25 days per year *4 years per period =426,612.  The hourly wage's variance is 207.70, which means the variance is $207.70*12^2*365.25^2*4^2=63,840,961,685.$
\footnote{
The hourly wage is calculated from the 2018 American Time Use Survey (ATUS), which asks respondents for their weekly earnings at their main job (coded by ATUS as "TRERNWA") and their weekly hours of work ("TEHRUSLT").  To calculate the hourly wage for a given individual, I divided TRERNWA by  TEHRUSLT.  The mean hourly wage was calculated by taking the average weighted hour of each individual's hourly wage, using ATUS's provided weight ("TUFINLWGT").  However, outliers in the bottom and top 2 percent were excluded.  Including the outliers would lift the mean hourly wage from \$24.42 to \$25.91.  Similarly, the variance of the hourly wage was calculated by taking the variance of each individual's hourly wage, again excluding the top and bottom 2 percent.  If the outliers were not excluded, the variance would rise from 207.70 to 505.81.}


%}  Multiplying this by 12 because there are 12 available hours in the day, times 365 times 4 because each period represents 4 years with 365 days each, we get that the mean wage is 24.35*12*365*4=426,612.  %TODO: explain how variance is calculated and check this whole section.

\end{itemize}




For most of the remaining calibrations, we will use data from Table \ref{table:items}.  Explanation of this data can be found in Appendix \ref{sec:data} %TODO




%\tiny
%\begin{table}
%
%\begin{tabular}{l| l l l l l l l l l}
%
%\toprule
% \textbf{\makecell{Item}} & \textbf{\makecell{Avg. Quantity}} & \textbf{\makecell{Avg. Expenditure}} & \textbf{\makecell{Avg. Time}} \\
%
%\midrule
%General Consumption ($c$)  & \$27,562 per year \tablefootnote{In 2018, the average annual consumer expenditures was \$61,224 per consumer unit, with 1.9 adults per unit.(\href{https://www.bls.gov/opub/reports/consumer-expenditures/2018/home.htm}{link})  Using 2017-2018 CPI-U weights, we know that 14.465 percent was spent on private transportation (\href{https://www.bls.gov/cpi/tables/relative-importance/2019.txt}{link})  Therefore, the amount of consumption, per adult, that was $not$ spent on private transportation is: \$61,224/1.9*(1-0.14465)= \$27,562.} & \$27,562 per year & N.A. \\  
%\hline
%Leisure ($L$)  & 5.62 hours per day \tablefootnote{Leisure is defined as time spent neither working nor driving.  Thus, leisure time is total time minus the driving time and labor time.  Total time is defined as 12 hours, because the person is assumed to start with 12 hours available for work, leisure, and driving.  (What of the remaining 12 hours of the day?  It is presumably spent sleeping, eating, in bathroom, etc.)  I have found the math and programming become faster and easier if I define the person as having a 12-hour-day than a 24-hour day.)}  & N.A. &   5.62 hours per day  \\
%\hline
%
%Driving ($m$) & 9,210 miles per year \tablefootnote{The total miles of private transportation, according to the 2017 National Highway Transportation , is 3,012,513 million person-miles.  This is the sum of the miles of various vehicles: (1,532,612 (car) +817,847 (SUV) + 260,856 (van) +386,559 (pickup truck) +461 (golfcart / segway) +9,676 (motorcycle / moped) +4,502 (RV) = 3,012,513 million miles) (\href{https://nhts.ornl.gov/person-miles}{link}).  The U.S. population in 2018 was 327,096,265 (\href{https://www.macrotrends.net/countries/USA/united-states/populations}{link}), so each person drove (or traveled via private transportation) 9210 miles per year.  This is because 301,2513 million/327,096,265=9210 miles per year.} & \$4,661 per year \tablefootnote{In 2018, the average annual consumer expenditures was \$61,224 per consumer unit, with 1.9 adults per unit.(\href{https://www.bls.gov/opub/reports/consumer-expenditures/2018/home.htm}{link})  Using 2017-2018 CPI-U weights, we know that 14.465 percent was spent on private transportation (\href{https://www.bls.gov/cpi/tables/relative-importance/2019.txt}{link})  Therefore, the amount of consumption, per adult, that was spent on private transportation is: \$61,224/1.9*0.14465= \$4,661.}& 45.57 minutes per day \tablefootnote{As described in the file helpful.r (Explain!), the 2018 American Time Use Survey shows that the average American spent 45.57 minutes per day driving (or engaged in private transportation). }\\
%\hline
%Labor ($l$) & 5.62 hours per day\tablefootnote{In 2018, according to ATUS data, the average American spent 5.62 hours per day working.  This estimate includes weekend days, and non-working adults (but not children).  It is the mean (and approximately the median of TEHRUSLT, see the file helpful.r)} &N.A. & 5.62 hours per day  \\
%
%\bottomrule
%\end{tabular}
%\caption{Quantities, Expenditures, and Time for $c$, $L$, $m$, and $l$}
%\label{table:items2}
%\end{table}



%
%\tiny
%\begin{table}
%
%\begin{tabular}{l| l}
%
%\toprule
% \textbf{\makecell{Item}} & \textbf{\makecell{Data (per adult)}} \\
%
%\midrule
%General Consumption ($c$)  & \makecell{\$27,562 spent per year on non-driving}\\  
%\hline
%Leisure ($L$)  & \makecell{5.62 hours per day} \\
%\hline
%Driving ($m$) & \makecell{9,210 miles per year\\ \$4,661 spent per year on all driving expenses\\ 45.57 minutes per day spent driving}\\
%\hline
%Labor ($l$) & \makecell{5.62 hours per day}\\
%\bottomrule
%\end{tabular}
%\caption{Quantities for $c$, $L$, $m$, and $l$.  Explanation of this data can be found in Appendix \ref{sec:data}.}
%\label{table:items}
%\end{table}
%
%\normalsize

\noindent
%\textcolor{red}{\Large\textbf{after}}
\begin{table}[!h]

\centering
\begin{threeparttable}


\begin{tabular}{l r}

\toprule
 {\makecell{Item}} & {\makecell{Data (per adult)}} \\

\midrule
\makecell[l]{General\\ Consumption ($c$)}  & \makecell[r]{\$27,562 spent per year on non-driving}\\  
\addlinespace
\midrule
\addlinespace
Leisure ($L$)  & \makecell[r]{5.62 hours per day} \\
\addlinespace
\midrule
\addlinespace
Driving ($m$) & \makecell[r]{9,210 miles per year,\\ \$4,661 spent per year on all driving expenses,\\ 45.57 minutes per day spent driving}\\
\addlinespace
\midrule
\addlinespace
Labor ($l$) & \makecell[r]{5.62 hours per day}\\
\bottomrule
\end{tabular}
\caption{Quantities for $c$, $L$, $m$, and $l$.\\  Explanation of this data can be found in Appendix \ref{sec:data}.}
\label{table:items}
\end{threeparttable}
\end{table}


\subsection{Price and Time-Cost of Driving} \label{sec:prices}
\begin{itemize}
\item $\boldsymbol{q}$ is calibrated to \textbf{\$510}, the price of driving 1,000 miles.  As Table \ref{table:items} shows, Americans spent (on average) \$4,661 on private transportation, and drove  9,210 miles per year.  This means each mile cost about \$0.51, or \$510 per 1000 miles.
\item $\boldsymbol{j}$ is calibrated to target the value of $h$ to be $\boldsymbol{1.71 * 10^{-3}}$.  As Table \ref{table:items} shows, Americans spent 45.57 minutes per day, or 277 hours per year, driving.  They drove 9,210 miles during this time, which implies each mile took 0.03 hours, or 1.8 minutes, on average.  Since each time-period in the model is 4 years, and each day in the model has 12 hours, each time period is 4*365*12=17,520 hours.  Each block of 1,000 miles took 0.03*1000=30 hours.  $30/17,520= 1.71 * 10^{-3}$, so $j$ must be calibrated so that $h$'s target is $1.71 * 10^{-3}$. % (TODO: Check accuracy of this section).
\end{itemize}  %(Note: ensure this is right!) #The target is either 0.00002748804714 or 0.00171232876712...better figure out which one!


\subsection{Elasticities of Substitution}

In this section I show how $\rho$ and $\delta$ - each one representing an elasticity of substitution (EOS) - are calibrated.  An EOS is a property of inputs to a utility function; it shows how the agent's optimal ratio between the inputs changes when the ratio of the prices change.  For example, if $\rho$, which is the EOS between leisure and driving distance, is high, then a large increase in the ratio of the price of leisure to the price of driving would cause the optimizing agent to greatly reduce the chosen ratio of leisure to driving distance.

To calculate $\rho$ and $\delta$, I used data from the 2018 American Time Use Survey (ATUS).  For each individual in the survey, I calculate how much time they spent in a private vehicle (by summing the value coded "TUACTDUR24" for the time they spent driving, being a passenger in a car, and riding in a taxi).  I also noted how much each individual works per day by taking their usual hours per week (coded as "TEHRUSLT").  Then, for each individual, I define their leisure time ($L_i$) as the time they spend per day neither in a private vehicle nor working.  Because I am using 12-hour days, this means I take 12 hours and subtract the time spent driving and the time spent working.  Furthermore, I calculate how much distance each person drove per day, $m_i$, by taking the time they spent driving and dividing it by $h$, using the value calibrated for $h$ in Section \ref{sec:prices}.  I also calculated the wage of each person, $w_i$, by dividing TRERNWA (the weekly wage) by TEHRUSLT (the usual weekly hours).  Thus, for each individual $i$ we have their wage $w_i$, their daily leisure hours $L_i$, and their daily miles driven $m_i$.  From Section \ref{sec:prices} we have a value for the monetary price of driving a mile, $q$, and the time-cost of driving a mile, $h$, which do not vary from person to person in this model.

Since $\rho$ represents the EOS between leisure and miles driven, we learn $\rho$ by seeing how a change in the ratio of their prices affects the ratio of leisure-to-miles-driven consumed.  For individual $i$ the price of leisure is $w_i$ because that is the wage foregone by not working during leisure.  The price of driving is $q+hw_i$, with the first term representing the monetary cost of driving one unit, and the second term representing the foregone wages by not working while driving.  Thus the ratio of the prices of leisure to driving is $\frac{w_i}{q+hw_i}$. 

I then get the value of $\rho$ by running this regression:

\begin{equation} \log(L_i/m_i) = \alpha_\rho + \rho  \log(w_i/(q+hw_i)+\epsilon_i\end{equation}

%I calculated $L_i$, which is the number of leisure hours a person 

%\begin{equation} \log(L/m) = \alpha_\rho + \rho  \log(p_L/p_m)\end{equation}

%where $L$ is the number of leisure hours, defined as the amount of time that people spent neither working nor driving.  $m$ is the number of hours people spent driving.  $p_L$ is the price (or opportunity cost) of leisure (that is, the person's hourly wage).  $p_m$ is the price (or opportunity cost) of driving a mile.  The opportunity cost of leisure is the wage that one earns at work, so $p_L=w_i$.  The opportunity cost of driving is the sum of the monetary cost of driving ($q$) plus the opportunity cost of not earning a wage during the driving time ($hw_i$).  Thus the regression becomes:

%\begin{equation} \log(L_i/m_i) = \alpha_\rho + \rho  \log(w_i/(q+hw_i)+\epsilon_i\end{equation}

%Note that the values of $q$ and $h$ do not vary from person to person in this model.  When running the model, I use the values of $q$ and $h$ described in Section \ref{sec:prices}.  But the values of $L_i$, $m_i$, and $w_i$ vary from person to person, and I run the above regression using data from the 2018 American Time Use Survey for those variables.  The resultant value of $\rho$ can be interpreted as the elasticity of substitution because the regression is a log regression, so $\rho$ represents the percent change in the $L/m$ ratio associated with a 1 percent change in the $p_L/p_m$ ratio.

\begin{itemize}
\item Using the above procedure, I calculated the value of $\boldsymbol{\rho}$ is $\boldsymbol{0.27}$, which implies that $L$ and $m$ are complements.%0.267824959767
\end{itemize}

$\delta$ represents the EOS between $c$ and $g$.  Recall that $c$ is an agent's non-driving consumption and $g$ is a consumption index representing consumption of both leisure ($L$) and driving ($m$), as described in Equation \ref{eq:g}.  To calibrate $\delta$ we must learn how the ratio of $c$ to $g$ changes as the ratio of prices between $c$ and $g$ change.  Specifically, we must perform the following regression:

\begin{equation} \log(c_i/g_i) = \alpha_\delta + \delta \log(p_c/p_{gi})+\epsilon_i \label{eq:regress_delta}
\end{equation}

Once again we use data from the 2018 American Time Use Survey.  For each person, $c_i$ is assumed to be 95 percent of their wage.  $p_c$ is always 1, by definition.
Following equation Equation \ref{eq:g}, we calculate $g_i$ for each person using this equation:

\begin{equation}g_i= \Bigg(\kappa_L (L_i)^\rho + (\kappa_m) (m_i)^\rho\Bigg)^\frac{1}{\rho}\end{equation}

using values of $L_i$ and $m_i$ from the data and using the calibrated values of $\kappa_L, \kappa_m$ described below and $\rho$ described above.  %TODO make that less awkward?  Also improve on consumption just being 95 percent of wages?

The price $p_g$ can be calculated using this formula: 
\begin{gather}\nonumber p_g =\\ 
\left(  \frac{\kappa_L}{\Bigg( \Big(w\Big)^\frac{\rho}{\rho-1} \Big( \frac {\kappa_L }{\kappa_L w} \Big) ^\frac{1}{\rho-1} + \Big(q\Big)^\frac{\rho}{\rho-1} \Big( \frac {\kappa_L }{\kappa_m w} \Big) ^\frac{1}{\rho-1}\Bigg)^\rho}+   \frac{\kappa_m}{\Bigg( \Big(w\Big)^\frac{\rho}{\rho-1} \Big( \frac {\kappa_m }{\kappa_L p_j} \Big) ^\frac{1}{\rho-1} + \Big(q\Big)^\frac{\rho}{\rho-1} \Big( \frac {\kappa_m }{\kappa_m q} \Big) ^\frac{1}{\rho-1}\Bigg)^\rho}\right)^\frac{-1}{\rho} \end{gather}

The exact derivation of this formula is described in Appendix \ref{sec:ces_calculation}, with $\kappa_L, \kappa_m$, and $\rho$ having values to their calibrations described elsewhere in this section.  

The intuition behind the formula for $p_g$ follows from an understanding of the properties of the CES utility function.  As shown in Appendix \ref{sec:ces_calculation}, it always costs the same amount of money to buy a util.  Since $g$ is defined as a CES utility function (with $L$ and $m$ as inputs), it follows that $g$ has a constant price, which is derived in Appendix \ref{sec:ces_calculation}.

\begin{itemize}
\item When we have calculated $c_i$, $g_i$, and $p_{gi}$ for each person, we can run the regression in Equation \ref{eq:regress_delta}.  I calculated the value of $\boldsymbol{\delta}$ is  $\boldsymbol{1.05}$, which implies that $c$ and $g$ are substitutes.%1.0479345702
\end{itemize}



%\begin{itemize}
%\item
% $\bm{\rho}$ is calculated to $\bm{0.267824959767}$, which implies that $L$ and $m$ are complements.  %-2.733781948
%\end{itemize}
%
%To calculate this, I performed the following regression, using data from the 2018 American Time Use Survey:
%
%$$ \log(L/m) = \alpha_\rho + \rho * \log(p_L/p_m)$$
%
%$L$ is the number of leisure hours, defined as the amount of time that people spent neither working nor driving.  $m$ is the number of hours people spent driving.  $p_L$ is the price (or opportunity cost) of leisure (that is, the person's hourly wage).  $p_m$ is the price (or opportunity cost) of driving a mile.  It includes both the direct monetary cost of driving, and the cost of not earning a wage during that time.  It is equal to $q+p_m*h$, with $q$ and $h$ being defined above (and constant for each individual) but $w$ varying for each person (and also equal to $p_L$ for that person).  Thus the same equation can be written as:
%
%$$ \log(L_i/m_i) = \alpha_\rho + \rho * \log(w_i/(q+h*w_i)+\epsilon_i$$
%\begin{itemize}
% \item $\bm{\delta}$ is calibrated to $\bm{1.0479345702}$, which implies that $c$ and $g$ are substsitutes.%$\0.04574194951$.
%To calculate this, i performed the following regression:
%$$ \log(c_i/g_i) = \alpha_\delta + \delta * \log(p_c/p_g)+\epsilon_i$$
%
%Where $c_i$ is a person's consumption, assumed for now to be 95 percent of a person's wage (TODO: this must be improved upon).  $g_i$ is defined as follows:
%
%$$g_i= \Bigg(\kappa_L (L_i)^\rho + (1-\kappa_L) (m_i)^\rho\Bigg)^\frac{1}{\rho} $$
%
%with $\rho$ equal to 0.268 as defined above.  $\kappa_L$ and $\kappa_m$ will be explained below.
%
%$p_c$ is equal to 1, and $p_g$ is equal to 
%
%$$\left(  \frac{\kappa_L}{\Bigg( \Big(w\Big)^\frac{\rho}{\rho-1} \Big( \frac {\kappa_L }{\kappa_L w} \Big) ^\frac{1}{\rho-1} + \Big(q\Big)^\frac{\rho}{\rho-1} \Big( \frac {\kappa_L }{\kappa_m w} \Big) ^\frac{1}{\rho-1}\Bigg)^\rho}+   \frac{\kappa_m}{\Bigg( \Big(w\Big)^\frac{\rho}{\rho-1} \Big( \frac {\kappa_m }{\kappa_L p_j} \Big) ^\frac{1}{\rho-1} + \Big(q\Big)^\frac{\rho}{\rho-1} \Big( \frac {\kappa_m }{\kappa_m q} \Big) ^\frac{1}{\rho-1}\Bigg)^\rho}\right)^\frac{-1}{\rho} $$
%
%The above equation will be explained in an appendix, to be added soon.
%\end{itemize}

\subsection{Share Parameters}
To calculate $\kappa_c$, $\kappa_L$, $\kappa_g$, and $\kappa_m$, we must first calculate the amount spent on $c$, $L$, $g$, and $m$, as a percentage of total spending.  Here, the spending must be equal to the total foregone opportunity cost (that is, both monetary and time-based opportunity costs).  These are shown in Table \ref{tab:opp}


%\begin{table}
%\begin{tiny}
%\begin{tabular}{l| l l l l l l  l}
%
%\toprule
% \textbf{\makecell{Item}} & \textbf{\makecell{Formula}} & \textbf{\makecell{Calculation}} & \textbf{\makecell{Value}} & \textbf{\makecell{Calibrated \\Quantity}} & \textbf{\makecell{Total Value}} & \textbf{\makecell{Percentage}}\\
%
%\midrule
%\makecell{General\\ Consumption ($c$)} & 1 & 1 &1 & \$27,562 & \$27,562 & 33\\
%\hline
%Leisure ($L$)  & $w$ & 106,653 & 106,653 & 5.62/12=0.47 & \$49,949 & 61\\
%\hline
%
%%Driving ($m$) & $q + hw$ & 510 + 0.00002748804714 * 106653 & 513 & 9.21 & \$4,724 & 6\\
%Driving ($m$) & $q + hw$ & $510 + 1.71 * 10^{-3} * 106,653$ & 513 & 9.21 & \$4,724 & 6\\
%
%\hline
%Total &N.A.&N.A.&N.A.&N.A.&82,235&100\\
%
%\bottomrule
%\end{tabular}
%\caption{Opportunity Cost}
%\label{tab:opp}
%\end{tiny}
%\end{table}%TODO can this table be improved?



\noindent
%\textcolor{red}{\Large\textbf{after}}
\begin{table}[!h]
\begin{small}
\centering
\begin{threeparttable}
%\begin{tabular}{@{} l S[table-format=7.0] l S[table-format=6.0] cc @{}} 
\begin{tabular}{@{} crrrrrr @{}}
\toprule


 {\makecell{Item}} & {\makecell{Formula}} & {\makecell{Calculation}} & {\makecell{Value}} & {\makecell{Calibrated \\Quantity}} & {\makecell{Total Value}} & {\makecell{Percentage}}\\

\addlinespace
\makecell{General\\ Consumption ($c$)} & 1 & 1 &1 & \$27,562 & \$27,562 & 33\\
\addlinespace
Leisure ($L$)  & $w$ & 106,653 & 106,653 & \makecell[r]{5.62/12\\=0.47} & \$49,949 & 61\\
\addlinespace

%Driving ($m$) & $q + hw$ & 510 + 0.00002748804714 * 106653 & 513 & 9.21 & \$4,724 & 6\\
Driving ($m$) & $q$+$hw$ & \makecell[r]{510+\\(1.71*$10^{-3}$)\\ *106,653} & 513 & 9.21 & \$4,724 & 6\\

\addlinespace
Total &N.A.&N.A.&N.A.&N.A.&\$82,235&100\\

\bottomrule
\end{tabular}
\caption{Opportunity Cost \\ The numbers in the Calibrated Quantity column come from Table \ref{table:items}.}
\label{tab:opp}
\end{threeparttable}

\end{small}
\end{table}%TODO can this table be improved?

%\normalsize


These numbers imply that $\kappa_c = 0.33$, $\kappa_g=0.61+0.06=0.67$,  $\kappa_L = 0.61/0.67=0.91$, and $\kappa_m=0.06/0.67=0.09$.
Implicityly, this method of calculating the $\kappa$s assumes that the utility function is Cobb-Douglas.  In fact $\delta$ and $\rho$ are not equal to 1 so the function is not perfectly Cobb-Douglas.  However, these estimates are suitable for the following reason: first, $\delta$ is close to 1 so the outer nest of the nested CES function is very close to being Cobb-Douglass.  $\rho$ is close enough to zero that the share-parameters have little result on the outcome, so the share-parameters being off is not important.\footnote{The CES function gives equal weights to its arguments as it approaches perfect complements, regardless of the value of share parameters.  When $\rho$ is close to zero, the agent will prefer equal amounts of $L$ and $m$ regardless of the values of $\kappa_L$ or $\kappa_m$.  For more on this (somewhat surprising) result, see page 86 of \citet{thoni}.   }


\subsection{Emissions Parameters}
\begin{itemize}
\item The emissions per mile traveled, $\boldsymbol{e_m}$ is calibrated to $\boldsymbol{4.02*10^{-4}}$.  According to the EPA, there are 4.640 metric tons CO\textsubscript{2}e released per year per gas-powered passenger vehicle, as of 2019.  Each vehicle travled 11,520 miles, on average, so that CO\textsubscript{2}e per mile is $4.640/11,520=4.02*10^{-4}$.  \citep{epa_calculator} % (TODO should this be per 1000?)  0.000402

\item The emissions per dollar spent on general consumption, $\boldsymbol{e_c}$, is $\boldsymbol{2.92*10^{-4}}$.  To calculate this, we divide total emissions (excluding that from driving) over total GDP (excluding spending on driving.)  Total emissions in 2018 is: $6.02 * 10^9$ CO\textsubscript{2}e \citep{climate}  %0.000292, 6020000000


We subtract emissions from driving, $4.02*10^{-4} * 9,210$ ($e_m$ times total miles driven).  

Total GDP is: $2.06*10^{13}$ in 2018 \citep{macrotrends_gdp}.  Total spending on driving is: $q$*total miles driven = 510/1,000*9,210.  Thus our calculation is $(6.02 * 10^9 - 4.02*10^{-4} * 9,210) / (2.06*10^{13} - (4.02*10^{-4}*9,210) = 2.92*10^{-4}$.%20611860000000

\item The emissions per dollar spent on government consumption, $\boldsymbol{e_g}$, is $\boldsymbol{2.50 *10^{-4}}$.  I was unable to find reliable data on emissions per dollar of government consumption.  In order to compute this number, I assumed that the government spends the same amount on driving as the general public.  %$\boldsymbol{0.000249876}$

Thus,

\begin{gather}\nonumber e_g =\\
\nonumber e_c * (1 - 1.45 * 10^{-1}) + e_m / q * (1.45 * 10^{-1}) = \\ 
\nonumber 2.92*10^{-4} * (1 - 1.45 * 10^{-1}) + 4.02*10^{-4} / 510 * (1.45 * 10^{-1}) = \\
2.50 * 10^{-4}\end{gather}

%\nonumber e_c * (1 - 0.14465) + e_m / q * (0.14465) = \\2.92*10^{-4} * (1 - 0.14465) + 4.02*10^{-4} / 510 * (0.14465) = \\
%2.50 * 10^{-4}\end{gather}  %(TODO: check if this is right?  it suggests government pollutes less than personal...maybe multiply by 1000 issue?)  %0.000249876

Studies have been done on how much part of the government - specifically, the military - emits.  In 2017, its budget was \$$6.47 * 10^{11}$ \citep{macrotrends_military} and its emissions were $5.9 * 10^7$ CO\textsubscript{2}e \citep{reuters}.  This suggests the emissions per dollar were $9.12 * 20^{-5}$, which seems to be a very low number.  However, the study was quite conservative, only counting emissions it was certain of.  Since this number is only for the military (not the whole government), and because it is likely not accurate, I will not use it in the calibrations. %\$646750000000   59000000  0.000091225
\end{itemize}












%\subsubsection{General Lessons}
%What lessons have we learned about the relationship between changes in labor to changes in pollution?
%
%--An uncompensated reduction in worker's wages will reduce (non-government) pollution in nearly every case we have considered, because it causes workers to produce and consume less goods.\par
%--To the extent that wage-reduction lowers the opportunity cost (price) of leisure, wage-reductions tend to increase leisure.  Thus, wage-reductions can lower pollution both by making agents generally poorer, and making them more likely to spend their endowment on leisure.\par
%--The one exception to this is the case in which driving and general consumption $c$ are substitutes to one another; in that case a switch toward the more-polluting good can cause an increase in pollution, despite the agent becoming generally poorer.\par
%--A decrease in the wage sometimes causes agents to work more, sometimes to work less.  This is often thought of as a battle between the substitution effect and the income effect.  (citation!) But even in the Cobb-Douglas case - in which theoretically those two effects cancel each other out - we see that a wage-reduction causes and increase in labor; this is due to the pecularity of this model (specifically, that one good has a monetary cost, another has a time-cost, and one good has both a money and a time cost.)\par
%--Our anaylsis suggests that labor rises, and labor lowerings, can each be accompanied by pollution increases or decreases, making the specific occurrence in each incident depending on the parameterizations.
%
%--The above lessons only apply to non-government pollution.  If the government spends the taxes on highly-pollution causing goods, then of course this can cause any of the cases we're considering to become more polluting.
%
%--And what if the government decides to rebate the tax collected to the agent?\par
% When utility functions are Cobb-Douglas or approximately Cobb-Douglas, this can lead to a large increase in leisure, because the agent is now spending not only $\kappa_L$ of the time-endowment on leisure, but also $\kappa_L$ of the rebate (a monetary endowment) on leisure.  Thus, when the utility function is Cobb-Douglas and the wage-tax is rebated, a large increase in leisure, and a large reduction in pollution, may result.  (With perfect complements, the tax-and-rebate scheme makes zero difference.  With perfect substitutes, a big difference can occur if a switch happens, but this is true regardless of if the tax money is rebated.) 


%\section{Model Validation}
%TODO: labor supply elasticity

\section{Model Validation}\label{sec:validation}

Our model's validity is dependent on its matching with key parameters in the data.  In particular, we are interested in matching estimates of labor supply elasticity, and driving's elasticity to both income and the price of driving.

Most studies of labor supply elasticity found that prime-age men had labor supply elasticities close to zero, implying that the only people who responded to wage changes were those on the fringes of the labor market, such as the second earner in two-earner households or older workers considering retirement.  For example, \citet{altug} and \citet{ziliak} are two studies that find prime-age married men provide labor with near-zero elasticity (Frisch elasticities of 0.14 and 0.16, respectively).\footnote
{The Frisch elasticity measures how people change their labor supply in response to expected wage-changes.  Economic theory generally predicts that the labor supply will shift more dramatically in response to expected than unexpected wage-shifts, so an estimate of Frisch elasticity serves as a ceiling on how high Marshallian elasticity can be.} 
According to \citet{keane}, these studies are flawed because they look only at the intensive margin (that is, they look at how the number of hours worked by an employed worker changes rather than seeing if wage-changes prompt people to switch between employment and non-employment). More recent papers that take into account the participatory margin find much larger Frisch elasticities.  \citet{erosa}, \citet{keane_wasi}, and \citet{iskhakov} find elasticies of 1.75, 0.74, and 1.5-1.8 among the male subgroups they study.\footnote
{\citet{keane} also argues that many studies underestimate prime-age male labor supply elasticity because they only consider how work-hours change in response to wage-changes, as opposed to considering how men change their education and training levels in response to wage-changes.}

\citet{wei} uses a dynamic model to estimate the price elasticity of gasoline, finding it is -0.2 in the short term and -0.5 in the long run.  \citet{bento} find elasticities between -0.25 and -0.30, and USDOE 1996 found -0.38.  \cite{berry} use Swedish household data and find the long-run fuel price elasticity of VMT (vehicle miles traveled) is -0.69 and the population's average income elasticity of VMT is 0.42.  Using odomoter readings of new vehicles registered in California and then later given a smog check, \citet{gillingham} finds a medium-run VMT fuel-price elasiticity for new cars of -0.22.%\footnote{\url{https://www.sciencedirect.com/science/article/abs/pii/S0166046213000653}}

In a revised version of this paper to be released shortly, I will show how well the model matches these data parameters. % TODO


\section{Results Explanation}\label{sec:results}


%\begin{table}\begin{tiny}
%%begin{tabular}{ | || | | | | | | || | |  }
%\begin{tabular}{l| l l l l l l l l l}
%%\begin{tabular}{lp{.1\linewidth} lp{.1\linewidth} lp{.05\linewidth} lp{.1\linewidth} lp{.1\linewidth} lp{.1\linewidth} lp{.1\linewidth} lp{.1\linewidth} lp{.1\linewidth} l}
%\toprule
%\textbf{Scenario}  & \textbf{\makecell{Avg. \\Consump-\\tion}} & \textbf{\makecell{Avg.\\ Leisure}} & \textbf{\makecell{Avg.\\ Miles\\ Driven \\(1000s)}} & \textbf{\makecell{Avg.\\ Labor}} & \textbf{\makecell{Avg.\\ Govt\\ Spending \\(or\\Rebate*)}}   & \textbf{\makecell{Private\\ Emissions}} &  \textbf{\makecell{Total\\ Emissions}} & \textbf{\makecell{Avg.\\ Wealth}}\\
%
%\midrule
%Baseline  & 304,622 	&0.41	 & 1.63	&0.59 &	0	&	87.04& 	87.04& 141,561 \\
%\hline
%Wage Tax ($\tau_w=.25$)  & 227,252 &	0.41	& 1.53	& 0.59&	69,737	&	65.06	&85.43& 106,171 \\
%\hline
%\makecell{Wage Tax ($\tau_w=.25$)\\ with Rebate} & 266,656& 0.5&	1.85&	0.5&	61,293*	&	76.36&	76.36 &126,506\\
%\hline
%Carbon Tax ($\tau_e=.882$) & 242,674&	0.41&	1.44&	0.59&	61,206	&	69.4&	87.27&141,561 \\
%\hline
%\makecell{Carbon Tax ($\tau_e=.882$)\\ with Rebate}  & 278975&	0.49&	1.7&	0.51&	70,377*&		79.79&	79.79&165,710 \\
%\hline
%Mandatory \\Retirement (y=15)   & 302,320&	0.45	&1.73&	0.55	&0	&	86.43&	86.43&226,060 \\
%\hline
%Work Limit (Y=.4)   & 199,447&	0.6&	1.97&	0.4&	0	&	57.35&	57.35 &66,581\\
%\hline
%
%
%\end{tabular}
%\caption{Experiments}
%\label{table:results}
%\end{tiny}
%\end{table}




\noindent

%\textcolor{red}{\Large\textbf{after}}
\begin{table}[!h] \begin{small}
\centering
\begin{threeparttable}
%\begin{tabular}{@{} l S[table-format=7.0] l S[table-format=6.0] cc @{}} 
\begin{tabular}{@{} c cccccccc @{}}

\toprule
%{nº} & {CID Ligando} & {Nombre Ligando} & {Afinidad} & \multicolumn{2}{c@{}}{RMSD}\\ 
%{nº} & {CID Ligando} & {Nombre Ligando} & {Afinidad} & \multicolumn{2}{c@{}}{RMSD}\\ 
{Scenario}  & {\makecell{Avg. \\Consump-\\tion}} & {\makecell{Avg.\\ Leisure}} & {\makecell{Avg.\\ Miles\\ Driven \\(1000s)}} & {\makecell{Avg.\\ Labor}} & {\makecell{Avg.\\ Govt\\ Spending \\(or\\Rebate*)}}   & \multicolumn{2}{c@{}}{Emissions} & {\makecell{Avg.\\ Wealth}}\\
\cmidrule(l){7-8}
& & & & & &Private &Total &\\
\midrule
Baseline  & 304,622 	&0.41	 & 1.63	&0.59 &	0	&	87.04& 	87.04& 141,561 \\
\addlinespace
\midrule
\addlinespace
Wage Tax ($\tau_w=.25$)  & 227,252 &	0.41	& 1.53	& 0.59&	69,737	&	65.06	&85.43& 106,171 \\
\addlinespace
\addlinespace
\addlinespace
\makecell{Wage Tax ($\tau_w=.25$)\\ with Rebate} & 266,656& 0.50&	1.85&	0.50&	61,293*	&	76.36&	76.36 &126,506\\
\addlinespace
\midrule
\addlinespace
Carbon Tax ($\tau_e=.882$) & 242,674&	0.41&	1.44&	0.59&	61206	&	69.4&	87.27&141,561 \\
\addlinespace
\addlinespace
\addlinespace
\makecell{Carbon Tax ($\tau_e=.882$)\\ with Rebate}  & 278,975&	0.49&	1.7&	0.51&	70,377*&		79.79&	79.79&165,710 \\
\addlinespace
\midrule
\addlinespace
\makecell{Mandatory \\Retirement (y=15)}   & 302,320&	0.45	&1.73&	0.55	&0	&	86.43&	86.43&226,060 \\
\addlinespace
\addlinespace
\addlinespace
Work Limit (Y=.4)   & 199,447&	0.60&	1.97&	0.40&	0	&	57.35&	57.35 &66,581\\
\bottomrule
\end{tabular}
\caption{Experiments}
\label{table:results}
\end{threeparttable}
\end{small}
\end{table}

%\normalsize

\subsection{Wage Tax}

Results are shown in Table \ref{table:results} on page \pageref{table:results}.

How did the addition of the wage-tax, $\tau_w$, cause the model's equilibrium to change?  Our goal is to compare the equilibrium in which there is no wage-tax ($\tau_w=0$) to that when $\tau_w = 0.25$.

First, the tax makes agents generally poorer.  So the tax has an income effect which, in isolation, would reduce $c$, $L$, and $m$ across all periods.  But what of the substitution effect?  How does the tax change the relative consumption of $c$, $L$, and $m$? 

To analyze the substitution effect, let us note that the price of $c$ is 1, $L$ has a price of $w(1-\tau_w)$ because choosing to spend time in leisure means that the opportunity to earn a wage is sacrificed, and $m$ has a price of $q+hw(1-\tau_w)$ because driving requires both monetary expenditure ($q$) and using time which also means sacrificing an opportunity to earn a wage.  (Here, I am temporarily ignoring the fact that the agent can choose to consume $c$, $L$, and $m$ in different periods, and that the price of consuming in each period is affected by the interest rate.  Thus the price of consuming $L$ in the nth period is $w/(1+r)^{n-1}$, not simply $w$, and similar for the prices of $c$ and $m$ across different periods.)

To briefly digress, note that agents have a nested CES utility function.  The variable $g$ represents the inner nest; it is an index representing consumption of leisure $L$ and driving $m$.  It is shown in Appendix \ref{sec:ces_calculation} that $g$ can be thought of as a good with a constant price $p_g$.  In other words, the agent chooses to buy some amount of $g$ each period; the expenditure on $g$ is equal to $gp_g$ is allocated among $L$ and $m$ for that period.

Now, back to analyzing the substitution effect of changing $\tau_w$ from 0 to 0.25.  The higher value of $\tau_w$ lowers the prices of $L$ and $m$ (and therefore $g$).  Faced with a wage-tax, agents buy relatively less $c$ and relatively more $g$.  Then, for the expenditures allocated to $g$, a further allocation is made between $L$ and $m$, with the ratio of $L$ to $m$ increasing because an increase in $\tau_w$ lowers the price of $L$ ($w$) by a greater proportion than it does the price of $m$ ($q+hw$).  

How large is the substitution effect?  We expect a 1 percent change in the ratio of the price of $c$ to the price of $g$ (in a given period) to change the ratio of optimal $c$ to $g$ by $\delta$ percent, as $\delta$ is the elasticity of substitution between $c$ and $g$.  We similarly expect a 1 percent change in the ratio of the price of $L$ and the price of $m$ (in a given period) to change the ratio of the chosen $L$ to $m$ by $\rho$ percent.

When we compare the total values for $c$, $L$, and $m$ consumed in two different equilibria - that with $\tau_w=0$ and that with $\tau_w=.25$, we see that the change in the ratios of $c$ and $g$, and $L$, and $m$, do nearly match the elasticities of substitution.  (Here, I calculated the changes in the prices of $L$, $m$, and $g$ using the average wage $w=471614$.)  Why did it not exactly match the elasticity of substitution?  One reason is that the value of $h$ changes slightly as $\tau$ changes; this is because the increased traffic causes it to take longer to drive a given distance.    Another is because these totals are sums of amounts consumed in different time periods, and also across agents with different productivities $w$. % (Note: can we modify the calculations to accomodate the change in h?)

Still the changes in the totals for $c$, $L$, and $m$ are remarkably close to that predicted by these calculations.  The same is true when we compare the results to similar calculations for the wage tax model with rebate, the carbon tax, and the carbon tax with rebate. 
%TODO Add a table showing what these results would look like?

\subsection{Understanding changes in labor and pollution}

Once we understand how changes in policy parameters, such as the wage-tax or emissions-tax, changes the values of $c$, $L$, and $m$, we can then see how such changes change the equilibrium values of labor and emissions.  In any given period, $l=1-L-mh$; therefore the amount of time worked is a side-effect of choosing the optimal values of $c$, $L$, and $m$.  Another way to think about this is that the agent - who takes their wage $w$, the interest rate $r$, and all taxes and government policies as given - must work and save enough to pay for all consumption $c$ and all driving $m$.  Thus the optimal values of $l$ are determined as the agent chooses other optimal values.

In this model, agents are not affected by pollution (unless the government mandates that emissions taxes be paid).  The amount of emissions does not affect utility.  Emissions are produced when agents choose to consume $c$ or drive $m$, or when the agents make decisions that cause the government to collect taxes and then spend it on polluting activities.

As we can see in Table \ref{table:results}, all of the experiments caused labor to either go down or remain unchanged, and pollution is always reduced or remains about the same.  Thus this paper suggests that, in general, policies that reduce labor will tend to not increase pollution, or not by much.  In other words, we conclude that the pollution-reducing effect of people producing less goods is NOT countered by people demanding more polluting goods (mostly).

The most instructive scenarios are those in which pollution is raised, or is reduced only slightly.  Consider the case in which the government collects an emission-taxes, which actually raised emissions from 87.04 to 87.27.  In that case, the government made $c$ and $m$ much more expensive, thus greatly reducing private (non-government) emissions, but the large amount of government-caused pollution actually led to an overall pollution increase.  Such a result suggests that using a tax on labor (or emissions) can be counter-productive (with regard to pollution-reduction) if the newly collected tax money is used in polluting activities.  Of course such a result is highly sensitive to how we calibrate $e_G$, the emissions per dollar of government spending.

It is also instructive to look at the scenario in which a retirement-age is introduced.  In this scenario pollution barely drops, despite a large drop in labor.  This is because agents know in advance that they are forced to retire early, and so they are able to save and plan to use their non-work time (the time when they are forced to retire) to drive.  Since driving is relatively polluting, the result is that the retirement-age equilibrium has almost as much pollution as the baseline equilibrium.

Of course, if the agents were forced to retire very early, they would not have time in their lives to work and save enough to spend their retirement driving.  This suggests that a small reduction in labor may sometimes be accompanied by pollution staying about the same, or even increasing.  

But a sufficiently large drop in labor will always lead to a reduction in pollution.  For example, if a person could not work at all, then nothing would be produced or consumed, and thus there would be no threat that the agent would switch to more polluting goods to counter out the pollution-reducing effects of a drop in labor.  This is illustrated in the final scenario, in which agents cannot work more than 0.4 each period.  That is such a large drop in labor that pollution drops dramatically.

\section{Conclusion}\label{sec:conclusion}

%The model in this paper focuses mainly on two mechanisms through which a drop in labor can affect pollution: first, the idea that a drop in labor leads to a drop in production/consumption, and therefore a drop in pollution.  Secondly, there is the idea that people change which goods and services they demand when they work less; this second effect can cause a drop in labor to cause either an increase or a decrease in pollution.  The overall effect of a drop in labor on pollution - that is, the net sum of these mechanisms - is at first ambiguous.\par

The model in this paper focuses mainly on two mechanisms through which a drop in labor can affect pollution: the reduced-production mechanism, and the reallocation-of-production mechanism.  Because the direction of the reallocation-of-production mechanism is at first unknown, we do not know \textit{a priori} if a reduction in labor will lead to an increase or a decrease in pollution.  

In this paper, I calibrated a model and ran experiments on it to see what the effect of drops in labor would be on pollution.  In every experiment I ran in which labor was reduced, pollution was also reduced.  That is, the reallocation-of-production mechanism didn't cancel out the reduced-production mechanism.

This finding is consistent with empirical studies of elasticities with regard to driving.  The elasticity of driving with regard to income is positive (for example, see \citet{berry}), so this suggests that people drive less as they make less money.  (Further research is needed to investigate the elasticity of driving with respect to working hours, as opposed to wages earned.)  

There are also additional mechanisms through which a drop in labor can affect pollution:

\begin{itemize}
\item \textit{Changes in productivity.}  As people work less, their productivity per hour may increase or decrease.  If less work leads to healthier or better-rested workers, this could increase productivity.  If less work leads to a decline in worker-experience, this could reduce productivity.  Holding constant the amount of pollution emitted from the production of a good or service, and assuming that amount is positive, an increase in productivity would lead to an increase in pollution per hour worked.  A decline in productivity woud lead to a decline in emissions per hour worked.

\item \textit{Changes in production techniques and capital usage.} As people work less, firms may shift their production techniques, perhaps by using more capital.  These new production techniques could lead to more or less pollution per hour worked.

\item \textit{Changes in pace of technological change.} As people work less, the pace of technological change may change.  Technological changes can introduce new production techniques for old goods and services, or introduce whole new goods and services.  Technological changes can be ``green'' (pollution-reducing), or they can lead to greater pollution.  Thus, either accelerations or deccelerations to the pace of technological change can be either pollution-increasing or pollution-decreasing.

\item \textit{Changes in composition of labor force.} Finally, any mechanism which induces people to work less probably will not cause the entire population to reduce their labor hours by the same proportion.  Some will drop their hours by a lot; some by a little or perhaps not at all.  Because workers vary in their skills and productivity, their labor also leads to different amounts of carbon emissions.  Thus a shift in the composition of the labor force can lead to change in the amount of carbon-emissions per hour of labor.

\item \textit{Changes in education.}  Any policy regime which affects people's incentive to work will also affect their incentive to get educated.  There are a large number of mechanisms through which education can affect pollution.  For example, if people spend less of their life getting educated, they may spend more time engaged in production.  Additionally, education may increase a person's productivity, so in that regard education may be positively correlated with pollution.  Education influences what technologies are available to a population; technological advances can sometimes lead to more pollution and sometimes less.  Education may also change a person's preferences over various combinations of goods, services, and leisure.

\end{itemize}

%Further research is needed .  \par

%In addition, the role of education ought to be incorporated as well.  Education can affect the labor-pollution channel in many ways: For example, education can affect what methods of production are available to a society, and it can affect people's preferences over various combinations of goods, services, and leisure.  (Under these mechanisms, becoming more educated can arguably cause more or less pollution.)  Perhaps most importantly, education can make people more productive workers.  In this paper, a more productive worker is -all else being equal- a more polluting worker, since a more productive worker can produce more stuff in a given unit of time, and this stuff is associated with carbon emissions.  \par

%It is worthwhile to consider what the effects would be if agents were induced to work less and study more.  If we assume studying causes little direct pollution, then a move from labor to study would reduce pollution.  However, if studying increases the productivity in agents, it could end up increasing pollution in the long term.  It would be worthwhile to study this issue in more detail, in the future.

Thus, further research is needed to develop a model that incorporates the numerous mechanisms through which a reduction in work can influence carbon emissions.


\newpage

\bibliography{ref_jmp}



\newpage
%\section*{Figures and Tables}
%
%\begin{figure}[H]
%\hspace{0.5in}
%	\includegraphics[height=3.2in,width=4.5in]{./PXL_20220923_222647980.PORTRAIT.jpg}%{./tax_func_holter.pdf}
%\caption{\scriptsize  The
%figure plots the tax functions for Germany and United States estimated in
%\cite{holter}.
%}\label{fig:fig5}
%\end{figure}
%\begin{figure}
%	\hspace{0.75in}
%	\includegraphics[height=3.2in,width=3.7in]{./PXL_20220923_222647980.PORTRAIT.jpg}%{./welfare_transition.pdf}
%	\caption{\setstretch{1.9} \scriptsize   The figure plots the welfare experienced by different
%		generations along the transition to a new tax regime. Welfare is
%		measured as the consumption equivalent variation
%		relative to the welfare of a generation born
%		with old tax regime. On the
%horizontal axis we identify generations by a number. Generations labeled -19 to
%0 are generations not affected by the policy. Generation labeled 1 experiences the
%tax function change during that
%generation's last year of life. The solid blue line plots the welfare of German
%workers under the German progressive tax system and the welfare they experience
%as they change to the less progressive U.S. tax system. The dash-dotted purple
%line plots the equivalent for
%U.S. workers as they experience the change to the more progressive German tax
%system. The two vertical dotted lines show the generation number 0 --
%the last generation not affected by the policy change -- and
%number 35 -- the first generation born under the new tax regime.}\label{fig:fig6}
%\end{figure}

\appendix

\section{Appendix: Data Explanation} \label{sec:data}
The purpose of this appendix is to explain the data in Table \ref{table:items}.

\subsection{General Consumption}
In 2018, the average annual consumer expenditures was \$61,224 per consumer unit, with 1.9 adults per unit. \citep{expenditures}  Using 2017-2018 CPI-U weights, we know that 14.465 percent was spent on private transportation \citep{cpi} Therefore, the amount of consumption, per adult, that was $not$ spent on private transportation is: \$61,224/1.9*(1-0.14465)= \$27,562.
%\href{https://www.bls.gov/cpi/tables/relative-importance/2019.txt}{link}
\subsection{Leisure}
Leisure is defined as time spent neither working nor driving.  Thus, leisure time is total time minus the driving time and labor time.  Total time is defined as 12 hours, because the person is assumed to start with 12 hours available for work, leisure, and driving.  (What of the remaining 12 hours of the day?  It is presumably spent sleeping, eating, in bathroom, etc.)  I have found the math and programming become faster and easier if I define the person as having a 12-hour-day than a 24-hour day.) %TODO fill this out more

\subsection{Driving}
\begin{itemize}
\item The total miles of private transportation, according to the 2017 National Highway Transportation , is 3,012,513 million person-miles.  This is the sum of the miles of various vehicles: (1,532,612 (car) +817,847 (SUV) + 260,856 (van) +386,559 (pickup truck) +461 (golfcart / segway) +9,676 (motorcycle / moped) +4,502 (RV) = 3,012,513 million miles) \citep{transportation}.  The U.S. population in 2018 was 327,096,265 \citep{macrotrends_population}, so each person drove (or traveled via private transportation) 9210 miles per year.  This is because 3,012,513 million/327,096,265=9,210 miles per year.
%TODO check links


\item In 2018, the average annual consumer expenditures was \$61,224 per consumer unit, with 1.9 adults per unit. \citep{expenditures}  Using 2017-2018 CPI-U weights, we know that 14.465 percent was spent on private transportation \citep{cpi}  Therefore, the amount of consumption, per adult, that was spent on private transportation is: \$61,224/1.9*0.14465= \$4,661.

\item  The 2018 American Time Use Survey shows that the average American spent 45.57 minutes per day driving (or engaged in private transportation).%TODO (Explain!  As described in the file helpful.r,  (Note...you need to convey the calculations done in helpful.r!!)
\end{itemize}


\subsection{Labor} In 2018, according to ATUS data, the average American spent 5.62 hours per day working.  This estimate includes weekend days, and non-working adults (but not children). % It is the mean (and approximately the median of TEHRUSLT, see the file helpful.r %TODO  This helpful.r info needs to be explained way better!!

\section{Appendix: The Price of a CES Util is Constant} \label{sec:ces_calculation}

In this section, I show that for an agent with budget $M$ and a CES utility function of N goods $x_1...x_N$ and prices $p_1...p_N$, each unit of utility costs a constant amount.  (In other words, if $M$ doubles, then the highest amount of utility the agent can achieve also doubles.)  This is relevant to this paper because it demonstrates that the variable $g$ in Equations \ref{eq:C} and \ref{eq:g}, which is an index measuring how much driving and leisure are consumed, can be thought of as a good in its own right, with a constant price of $p_g$.

First, let's describe the agent's problem.

\begin{equation}\max_{x_1...x_N}   \Bigg(\sum_{i=1}^N \alpha_i x_i^\rho\Bigg)^\frac{1}{\rho} such that  \sum_{i=1}^N x_i p_i \le M\end{equation}

%Solve:
%$$\Bigg(\sum_{i=1}^N \alpha_i x_i^\rho\Bigg)^\frac{1}{\rho} + \lambda(M-\sum_{i=1}^N x_i p_i )$$

\subsection{Solving the Demand Function}

Now let's solve for the demand function.
FOC (for each $x_j$):
\begin{equation}\frac{1}{\rho}\Bigg(\sum_{i=1}^N \alpha_i x_i^\rho\Bigg)^\frac{1 - \rho}{\rho} \rho \alpha_j x_j^{\rho-1} = \lambda p_j\end{equation}
%FOC (for $x_1$):
%$$\frac{1}{\rho}\Bigg(\sum_{i=1}^N \alpha_1 x_i^\rho\Bigg)^\frac{1 - \rho}{\rho} \rho \alpha_1 x_1^{\rho-1} = \lambda p_1$$
Divide FOC for $x_i$ by FOC for $x_j$:
\begin{equation}\frac{\alpha_i}{\alpha_j}\Bigg(\frac{x_i}{x_j}\Bigg)^{\rho-1} = \frac {p_i}{p_j}\end{equation}
Therefore:
\begin{equation}\frac{x_i}{x_j} = \Bigg( \frac {\alpha_j p_i}{\alpha_i p_j} \Bigg) ^\frac{1}{\rho-1}\end{equation}
Budget constraint:
\begin{equation}M =\sum_{i=1}^N x_i p_i\end{equation}
Divide by $x_j$:
\begin{equation}\frac{M}{x_j} =\sum_{i=1}^N p_i\frac{x_i}{x_j}\end{equation}
Substitute:
\begin{equation}\frac{M}{x_j} =\sum_{i=1}^N p_i \Bigg( \frac {\alpha_j p_i}{\alpha_i p_j} \Bigg) ^\frac{1}{\rho-1}  =\sum_{i=1}^N \Bigg(p_i\Bigg)^\frac{\rho}{\rho-1} \Bigg( \frac {\alpha_j }{\alpha_i p_j} \Bigg) ^\frac{1}{\rho-1}\end{equation}
Therefore:
\begin{equation}x_j^* =\frac{M}{\sum_{i=1}^N \Bigg(p_i\Bigg)^\frac{\rho}{\rho-1} \Bigg( \frac {\alpha_j }{\alpha_i p_j} \Bigg) ^\frac{1}{\rho-1}}\end{equation}

\subsection{Utility Achieved in Equilibrium}

Now let's determine the utility achieved in equilibrium.
\begin{equation}U = \Bigg(\sum_{j=1}^N \alpha_j x_j^\rho\Bigg)^\frac{1}{\rho}\end{equation}

We sub in the values of $x_j^*$.
\begin{equation}U = \left(\sum_{j=1}^N \alpha_j \Bigg(\frac{M}{\sum_{i=1}^N \Big(p_i\Big)^\frac{\rho}{\rho-1} \Big( \frac {\alpha_j }{\alpha_i p_j} \Big) ^\frac{1}{\rho-1}}\Bigg)^\rho\right)^\frac{1}{\rho} \end{equation}
\begin{equation}U =  \left(M^\rho \sum_{j=1}^N  \frac{\alpha_j}{\Bigg(\sum_{i=1}^N \Big(p_i\Big)^\frac{\rho}{\rho-1} \Big( \frac {\alpha_j }{\alpha_i p_j} \Big) ^\frac{1}{\rho-1}\Bigg)^\rho}\right)^\frac{1}{\rho} \end{equation}
\begin{equation}U =  M \left( \sum_{j=1}^N  \frac{\alpha_j}{\Bigg(\sum_{i=1}^N \Big(p_i\Big)^\frac{\rho}{\rho-1} \Big( \frac {\alpha_j }{\alpha_i p_j} \Big) ^\frac{1}{\rho-1}\Bigg)^\rho}\right)^\frac{1}{\rho} \end{equation}

Therefore, the price of one unit of utility is:
\begin{equation}
\left( \sum_{j=1}^N  \frac{\alpha_j}{\Bigg(\sum_{i=1}^N \Big(p_i\Big)^\frac{\rho}{\rho-1} \Big( \frac {\alpha_j }{\alpha_i p_j} \Big) ^\frac{1}{\rho-1}\Bigg)^\rho}\right)^\frac{-1}{\rho} 
\label{eq:price_of_util}
\end{equation}
The price of one unit of utility is always constant (that is, it always based on values which the agent takes as given).

\subsection{Why does this matter?}

In our model, we used a nested utility CES function.  The outer nest is described by Equation \ref{eq:C}.  That equation contains the variable $g$, which is itself a CES function; it is the inner nest and described in Equation \ref{eq:g}.  As shown above, for CES utility functions, utils can be bought at a constant price.  Therefore, we can think of $g$ as having a constant price $p_g$.  That constant price is described in \ref{eq:price_of_util}, keeping in mind that the two ($N=2$) goods are Leisure (with price $p_1= w$) and driving (with price $p_2 = q+hw$).



\section{Appendix: Simple Model}\label{sec:simplified}

To understand the paper's main model - described in Section \ref{sec:model}, it is helpful to think about a much simpler problem.
Suppose a single agent, with wage $w$ solves this one-period problem:

\begin{equation}\max_{c, L, m, l} U\end{equation}

such that

\begin{equation}c + mq  = lw\end{equation}
\begin{equation}L + mh + l = 1\end{equation}

The utility function is defined as follows:

\begin{equation}U=
   \begin{cases} 
       \Bigg(\kappa_c c^\frac{\delta-1}{\delta} + \kappa_g g^\frac{\delta-1}{\delta} \Bigg)^\frac{\delta}{\delta-1}& \text{for } \delta \ne 1 \\
      \\
     c^{\kappa_c}g^{\kappa_g} & \text{for } \delta = 1\\ 
   \end{cases}
\end{equation}

\begin{equation}g=
   \begin{cases} 
       \Bigg(\kappa_L L^\frac{\rho-1}{\rho} + \kappa_m m^\frac{\rho-1}{\rho} \Bigg)^\frac{\rho}{\rho-1}& \text{for } \rho \ne 1 \\
      \\
     L^{\kappa_L}m^{\kappa_m} & \text{for } \rho = 1\\ 
   \end{cases}
\end{equation}

The choice variables are consumption $c$, leisure $L$, driving miles $m$, and labor $l$.  The market price of driving is $q$.  The time it takes to drive one unit is $h$.  $q$, $h$, and the wage $w$ are all taken as given.  The parameters of the utility function ($\delta$, $\rho$, $\kappa_c$, $\kappa_g$, $\kappa_L$, and $\kappa_m$) are exogenous, with $\kappa_c+\kappa_g = \kappa_L+\kappa_m=1$.

The last part of the model is the amount of carbon emissions $E$:

\begin{equation}E = ce_c + me_m\label{eq:emissions}\end{equation}

where $e_c$ and $e_m$ are the exogenous amounts of emissions per unit of consumption and driving, respectively.

\subsection{Simplifying the Simple Model}

We can simply the simple model further by combining the two constraints, the budget constraint and the time constraint, into one:
\begin{equation}c+Lw +m(q+hw) =w\label{eq:budget}\end{equation}

Seen this way, we can see that $c$, $L$, and $m$ can be thought of as goods with the following prices: 1, $w$, and $q+hw$, and the agent has budget $w$ to spend on those three goods.  So we can think of the problem as:

\begin{equation}\max_{c, L, m} U\label{eq:optim}\end{equation}

such that

\begin{equation}c+Lw +m(q+hw) =w\end{equation}

In this version, $l$ is not being explicitly optimized for, but the optimal value of $l$ is defined in terms of the other optimal values:

\begin{equation}l^*=1-L^*-m^*h=\frac{c^*+m^*q}{w}\label{eq:labor}\end{equation}

The carbon emissions $E$ continue to be a value that plays no role in the optimization problem (equation \ref{eq:optim}), but can be calculated using equation \ref{eq:emissions} once the optimization problem has been solved, just as $l$ can be calculated using \ref{eq:labor} after the optimization problem has been solved.

\subsection{Viewing the Problem as a Two-Step Problem}

Now, suppose that $M \le w$ is the optimal amount that the agent spends on goods $L$ and $m$, with the rest being spent on $c$.

\begin{equation}L^*w+ m^*(q+hw)=M\end{equation}

\begin{equation}c^*=w-M\end{equation}

The optimal values of $L$ and $m$ in the above problem are also optimal in the following sub-problem:

\begin{equation}\max_{L, m} g\end{equation}
such that
\begin{equation}Lw+ m(q+hw)=M\end{equation}

Furthermore, it can be shown (see Appendix \ref{sec:ces_calculation})  that when the optimal values of $L$ and $m$ are chosen, $g=\frac{M}{p_g}$, where $p_g$ is a constant (in the sense that it depends only on values which the agent takes as given.)  Therefore, $p_g$ can be thought of as the price of one unit of $g$.

Or, to put it another way, we can think of the agent as solving first this problem:
\begin{equation}\max_{c, g} U\end{equation}

such that

\begin{equation}c + gp_g=w\end{equation}

And then, once the values of $c$ and $g$ have been chosen, solving this problem:

\begin{equation}\max_{L, m} g\end{equation}

such that

\begin{equation}Lw + m(q+hw) = M = gp_g\end{equation}

\subsection{The effect of a change in $w$ on labor and pollution}

Labor $l$ and emissions $E$ are not directly part of the optimization problem described in equation \ref{eq:optim}.  To understand the effects of a change in $w$, we must consider first how a change in $w$ causes a change in the optimal values of $c$, $L$, and $m$.  We then use equations \ref{eq:labor} and \ref{eq:emissions} to see how the values of $l$ and $E$ are changed.

In the budget constraint (equation \ref{eq:budget}) we see that $w$ appears three times: first as the price of $L$, second as part of the price of $m$ ($q+hw$), and finally on the right side of the equation as the agent's endowment.  Furthermore, $p_g$ (the price of $g$) depends on the prices of $L$ and $m$, so $p_g$ itself depends on $w$.  

Let us first consider that when $w$ changes, the agent's endowment changes.  Because the agent has homothetic preferences, a change in the agent's endowment (holding all else constant) does not change the ratios of the optimal $c$, $L$, and $m$ to each other.  An increase in the endowment - holding prices equal - causes the agent's optimal values of $c$, $L$, and $m$ to change proportionately.

However, a change in $w$ represents a change in the endowment as well as changes in the relative prices of $c$, $L$, and $m$, as equation \ref{eq:budget} makes clear.  Changes to these relative prices changes the optimal ratios of $c$, $L$, and $m$.

Specifically, a change in the ratio of the price of $c$ to the price of $g$ will change the optimal ratio of $g$ to $c$ by $\delta$.  And a change in the ratio of the price of $L$ to the price of $m$ will change the optimal ratio of $m$ to $L$ by $\rho$. 


Therefore, when $w$ changes, the ratio of the prices of $c$ and $g$ change, which means the optimal ratio of $c$ and $g$ will change, as determined by their elasticity of substitution $\delta$.  The new optimal values of $c$ and $g$ will satisfy this new ratio, while also satisfying the new budget constraint $c+gp_g=w$.  The optimal ratio of $L$ and $m$ will similarly change according to their new prices and their elasticity of substitution $\rho$; the actual new values of $L$ and $m$ must satisfy the new optimal ratio and also satisfy the budget constraint $Lw+m(q+hw)=M=gp_g$.  And then the changes to $l$ and $E$ are determined by equations \ref{eq:labor} and \ref{eq:emissions}.

%In this model, it is useful to think of labor $l$ and emissions $E$ changing as side effects to the changes in $c$, $L$, and $m$.  The agent doesn't directly lose utility from work or from pollution.  But labor and emissions are determined as follows:

\subsubsection{Wage decrease in the Cobb-Douglas case}

Let us consider a wage-change in the case of an agent who has a simple Cobb-Douglas utility function:

\begin{equation}U=c^\frac{1}{3}L^\frac{1}{3}m^\frac{1}{3}\end{equation}

and must choose $c$, $L$, and $m$ such that $c+Lw+m(q+hw)=w$.

The optimizing agent chooses $c=w/3$ and $L=1/3$.  This is another way of saying that the agent spends one-third of their time working to buy the consumption good $c$ and one-third of their time in leisure.  The remaining one-third of time is dedicated toward driving, with some division of that time between working in order to pay the monetary cost of driving, $q$, and some time spent actually driving.  In other words, $\frac{mq}{w}+mh = \frac{1}{3}$.  Solving for $m$, we get $m=\frac{1}{3(\frac{q}{w}+h)}$

Therefore the agent's total working time is $l=1/3 + \frac{q}{3(q+hw)}$, with the first term being the time spent working to pay for $c$, and the second term being time spent working to pay for $m$.  Notice that as $w$ decreases, the agent will work more, consume less, have the same amount of leisure, and drive less.  Therefore, in this simple Cobb-Douglas case, a wage-decrease will cause the agent to work more and pollute less.

\subsubsection{Wage decrease in the perfect complement case}

If the utility function was a simple complement function, $U=\min(c, L, m)$, then $c=L=m=\frac{w}{1+w+q+hw}$ and $l=\frac{1+q}{1+w+q+hw}$.  Thus a wage-decrease would also cause the agent to work more, consume less, drive less, and pollute less.  (But, unlike in the Cobb-Douglas case, the agent will also have less lesiure time.)

\subsubsection{Wage decrease in the perfect substitute case}

If the utility function was a simple substitute function, $U=c+L+m$, there are many cases to consider.  The wage-decrease may induce the agent to continue consuming only $c$, continue consuming only $L$,  continue consuming only $m$, switch from consuming only $c$ to only $m$, switch from $c$ to $L$, or switch from $m$ to $L$.  In the first case, work will stay the same and pollution will go down.  In the second case, work and pollution will stay at zero.  In the third case, work will increase and pollution will go down.  In the fourth case, work will go down and pollution may go up (depending on the values of $e_c$ and $e_m$).  In the fifth and sixth case, both work and pollution will go down to zero.


\subsection{Wage decrease vs wage-tax increase}
In the above examples, we have considered the effects of a decrease in wages.  In the main part of this paper, I consider the effects of implementing a wage tax.  It is tempting to equate a wage-decrease with an increase in the wage-tax, but the two are not equivalent, as the ultimate effect of a wage-tax also depends on how the taxes are spent by the government.  In the main part of the paper, I consider the case in which the wage-tax is spent by the government (in ways that cause emissions) and the case in which the wage-taxes fund an equal transfer to all living agents.

\section{Appendix: Dynamics}

We now study a dynamic version of the model.  The original setup of this version is exactly the same as the original version, described in the previous chapter.  The difference occurs at a certain time $\xi$: the government unexpectedly announces an immediate change to a policy variable.  For example, it may change the wage tax from an original value of $\tau_w=0$ to $\tau_w=0.25$.  We assume that before the change is announced, the economy is in a steady-state equilibrium in which the original policy ($\tau_w=0$ in this example) is expected to last forever.  At time $\xi$ and after, all agents expect the new policy ($\tau_w=0.25$ in this example) to last forever.\par

In order to solve this model, we have two goals:\par
\begin{enumerate}
\item To solve for the original steady-state that exists before the policy change is announced and enacted.  (This was described in the previous chapter.)
\item To solve for the time-path of all variables starting at date $\xi$.  (This is the topic of this chapter.)
\end{enumerate}

To see how agents behave at date $\xi$, we must know how much wealth they have at date $\xi$. \par

\subsection{An Agent's Wealth at Time $\xi$}

An agent's wealth at a certain age can be calculated based on how much wealth they had in the last period, adjusted for how much new wealth was gained or lost during that period.  Thus:

\begin{equation}D_{a+1} = (D_a + w l_a(1-\tau_w)+T_{b+a} - c_a(1+\tau_e e_c) - m_a(q+\tau_e e_m))  (1 + r)\end{equation}

with $D_0=0$, which means that each agent is born with zero wealth.\par

Thus, once we have calculated how agents behave in the original steady-state, we can calculate how much wealth a certain agent has at the time $\xi$ when the policy-change is enacted, based on how old they are at time $\xi$ and their wage $w$.\par
For a given agent, we will use $A$ to represent the agent's age at time $t=\xi$.  Thus $D_A$ is the wealth that the agent is holding at the time when the new policy-change is announcted and enacted.

\subsection{The Dynamic Version of the Utility Function and the Agent's Problem}

The dynamic version of the agent's problem is as follows:\par
At age $a=A$ and time $t=\xi$, an agent who was born in period $b$ and has unchanging wage $w$ is holding wealth $D_A$. The agent solves the following problem:

\begin{equation}\max_{\substack{c_A,...,c_{N-1},\\ L_A,...,L_{N-1},\\m_A,...,m_{N-1},\\ l_A,...,l_{N-1}}} U\end{equation}

Note that this problem is almost identical to that described in the last chapter, but now the agent is solving the problem at age $a=A$ instead of $a=0$.  Furthermore, the utility function is now:

\begin{equation}U = \sum_{a=A}^{{N-1}} \beta^{a-1} u\Big(c_{a}, L_{a}, m_{a}\Big)\end{equation}

Again, this reflects that the agent is solving the problem at age $a=A$ instead of $a=0$.  The period utility function and all its components are exactly as described earlier.\par

Furthermore, the agent is again subject to the time constraint described earlier, and can be subject to the retirement and work-maximum constraints described earlier if the government has imposed them.  The agent is subject to a budget constraint which is slightly different from the one described earlier:

\begin{equation}\sum_{a=A}^{N-1} \frac{D_A + wl_{a, b+a} (1-\tau_w)+T_{b+a}}{(1+r)^{a-A}}  \ge \sum_{a=0}^{N-1} \frac{c_{a, b+a}(1+\tau_e e_c)+m_{a, b+a}(q+\tau_e e_m)}{(1+r)^{a-A}}\end{equation}

This constraint reflects that the agent begins with wealth $D_A$ and that the agent is solving the problem at age $a=A$ rather than age $a=0$.  It is otherwise identical to the budget constraint described earlier.









\subsection{Dynamic Equilibrium (DE)}
Intuitively, a model-economy is in dynamic equilibrium if, starting from time $\xi$ to $\infty$, all agents make optimal choices consistent with the time-path of all aggregate variables, and the time-paths of all aggregate variables are consistent with the optimal choices of all agents.  Now I will give a specific definition.\par


Let $x_{w,a,t}$ be a quadruple $(c_{w,a,t}, L_{w,a,t}, m_{w,a,t}, l_{w,a,t})$.\par
A DE-candidate is a set of:\\
\begin{enumerate}
\item $x_{w, a, t}$ for all $w$ with support in $f(w)$, and all ages $a$ in $\{0,..., N-1\}$, and all $t$ in $\{\xi,...\infty\}$.\\
\item $h_t$ for all $t$ in $\{\xi,...\infty\}$.\\
\item $T_t$ for all $t$ in $\{\xi,...\infty\}$.
\end{enumerate}

A DE-candidate is a DE if and only if:
 \begin{enumerate}

\item Given the set of $h_t$ and $T_t$, $x_{w,a,t}$ are the optimal solutions to the agent's problem for all $t$ in $\{\xi,...\infty\}$.
\item Given the value of $m_{w,a,t}$ in the set of $x_{w,a,t}$, the values of $h_t$ must be equal to the value calculated using Equations \ref{eq:Ad} and \ref{eq:hd}.
\item If taxes are not being refunded, then $T_t=0$ for all $t$ in $\{\xi,...\infty\}$\\
\item If taxes are being refunded, then $T_t = A^\tau_t$ for all $t$ in $\{\xi,...\infty\}$, with $A^\tau_t$ being calculated using Equations \ref{eq:As} and the values of $x_{w, a, t}$
\end{enumerate}



\subsection{Discussion of Dynamic Results}
\subsubsection{Baseline to Wage Tax, No Rebate}

Let us first discuss the following case.  The economy starts out in a Baseline situation in which the government has enacted no policies.  In other words, there is no wage tax, emissions tax, mandatory retirement, or work maximum.  All agents expect Baseline these policies (that is, the lack of policies) to last forever.\par

Then, the government imposes a wage tax, changing the value of $\tau_w$ from 0 to 0.25, and the government spends all the tax revenue collected.  This new policy is enacted as soon as it is announced, and all agents immediately start expecting this new policy to last forever.  The bottom left corner of Figure \ref{figure:wage_tax_dyn} shows how aggregate wealth changes after the new policy is enacted.  (In all figures, the dotted line shows period $t=\tau$, when the new policy is announced and enacted.  To the left of the dotted line is the Baseline steady-state; to the right of the dotted lines show the dynamic paths under different policies.  In the following paragraphs, I will be discussing the grey curves to the right of the dotted line in Figure \ref{figure:wage_tax_dyn}.)  \par




As shown in Table \ref{table:results}, the average wealth is 141,561 in the Baseline steady state, and is 106,171 in the steady-state in which the government has imposed a wage tax of $\tau_w=0.25$, and the government spends the tax money it collects.  The bottom left corner of Figure \ref{figure:wage_tax_dyn} shows that wealth starts at level 141,561 because the economy starts out in the Baseline steady-state.  Then, after the new tax is enacted, the level of wealth gradually declines toward the new steady-state level of 106,171.\par

Immediately after the policy is announced, aggregate consumption jumps.  But it does not jump straight to the new steady-state value (227,252); it jumps to a value greater than this value.  Aggregate consumption then declines to its new steady-state value. \par 

Thus aggregate wealth is a state varible; aggregate consumption is a jump variable.  Other jump variables include aggregate leisure, aggregate driving, and aggregate emissions.  For all these variables, there is a jump as soon as the new policy is enacted; these variables then gradually decline toward their new steady-state values.\par

Aggregate labor is also a jump variable; it jumps as soon as the new policy policy is announced, then $increases$ towards its new steady-state value (0.59).\par

This should all match with a simple intuition:  
\begin{enumerate}
\item Aggregate wealth, being a state variable, gradually moves from its old steady-state value toward the new steady-state value.

\item  Aggregate consumption, leisure, driving, and emissions gradually go down as wealth is spent down.  But they first jump, as soon as the new policy is announced, to reflect the new circumstances; they then gradually go down toward their new steady-state values.  And similarly aggregate labor jumps as soon as the new circumstances are announced, then gradually increases as the agents spend down their wealth.
\end{enumerate}








\begin{figure}[H]
\centering
\includegraphics[width=175mm]{dynamic_figures/my_plots_wage_tax_rebate.pdf}
\caption{Dynamics of transition from Baseline to a wage tax ($\tau_w=0.25$).  To the left of the dotted line, the economy is in a steady-state in which Baseline policies are expected to last forever.  The dotted line is at the period ($\xi$) when the new wage tax is announced and enacted (and then expected to last forever).  The grey path shows the dynamics if the government spends all tax money collected; the black path shows the dynamics if the government rebates all the tax money.}
\label{figure:wage_tax_dyn}
\end{figure}

\begin{figure}[H]
\centering
\includegraphics[width=175mm]{dynamic_figures/my_plots_carbon_tax_rebate.pdf}
\caption{Dynamics of transition from Baseline to an emissions tax ($\tau_e=0.882$).  To the left of the dotted line, the economy is in a steady-state in which Baseline policies are expected to last forever.  The dotted line is at the period ($\xi$) when the new emissions tax is announced and enacted (and then expected to last forever).  The grey path shows the dynamics if the government spends all tax money collected; the black path shows the dynamics if the government rebates all the tax money.}
\label{figure:emissions_tax_dyn}
\end{figure}

\begin{figure}[H]
\centering
\includegraphics[width=175mm]{dynamic_figures/my_plots_work_max.pdf}
\caption{Dynamics of transition from Baseline to policies limiting work.  To the left of the dotted line, the economy is in a steady-state in which Baseline policies are expected to last forever.  The dotted line is at the period ($\xi$) when the new policy is announced and enacted (and then expected to last forever).  The grey path shows the dynamics if the government enforces a retirement age ($y=15$); the black path shows the dynamics if the government limits how much an agent can work in a given period ($Y=0.4$).}
\label{figure:work_limit_dyn}
\end{figure}


\subsubsection{Other Scenarios}

The above discussions show how aggregate variables change when we move from the Baseline Scenario (with $\tau_w=0$) to one with $\tau_w=0.25$, and the government spends rather than rebates the money.   We found that wealth is gradually spent down, and that other variables - such as consumption, leisure, and driving -initially jump and then gradually go down as the wealth is spent down.\par

I find similar results for other scenarios.  I looked at moving from the Baseline Scenario to a wage tax with the tax money being rebated; from Baseline to an emissions tax (with and without the tax money being rebated), from Baseline to a mandatory retirement age, and from Baseline to a work maximums.  In each case, aggregate wealth gradually moves from its old steady-state value to its new steady-state value.  If wealth moves down, then consumption, leisure, driving, and emissions go down as wealth is spent down, but labor moves up.  If wealth moves up, then consumption, leisure, driving and emissions go up as wealth is saved up, but labor goes down.  But in all cases, the jump variables - consumpion, leisure, driving, emissions, and labor - first jump before gradually heading toward their steady-state values described in Table \ref{table:results}.\par

In fact, in one case - the case of moving from Baseline to an emissions tax with rebate, the new steady-state value of aggregate wealth is nearly the same as the old steady-state value of aggregate wealth. In this case the jump variables jump straight to their new values, because wealth is not changing at all.

\subsubsection{Relationship between Labor and Pollution}

Our experiment shows a situation in which labor is increasing while pollution is decreasing.  For example, Figure \ref{figure:wage_tax_dyn} shows labor gradually increasing (after a jump), and Figure \ref{figure:wage_tax_dyn} shows emissions decreasing (after a jump), both during the time periods after a new wage tax has been annouced.  (This is true whether the government spends or rebates the tax, although the jump in emissions is visibly small in the case when the government spends the tax revenue directly.)  Why does emissions go down as labor goes up?  It is not because the labor is becoming less polluting.  Rather, it is because the dynamic model features a temporary detachment between labor and emissions.  The emissions are resulting from consumption and driving that are financed by a spenddown in wealth; this wealth was labored for in previous periods.  Therefore the seeming detachment between labor and emissions is a temporary phenomenon that persists as the economy moves from one steady-state to another.  (One can imagine that the wealth is being paid to foreigners - that is, agents outside the model - who are performing the necessary labor.)



\subsection{Summary of Dynamics}

In this paper, we studied the behavior of aggregate variables when the economy is originally in a base-line steady-state, and then a surprise policy change is simultaneously announced and enacted.  The new policy regime is then expected to last forever.  What we found is:

\begin{enumerate}
\item  The model begins in the original steady-state.  

\item On the date that the new policy is announced and enacted, aggregate variables (aggregate consumption, aggregate leisure, etc.) jump.

\item  Then, there is a transition in which the aggregate values gradually head toward a new steady-state.

\item Eventually they end up in a new steady-state, which then goes on for infinite periods.  

\item The values of the aggregate variables, in both the original and the new steady state, are explained in a previous chapter of this dissertation.  Here we are interested in the transition between the two steady-states.

\item  To understand the transition between the two steady-states, it is important to look at the aggregate wealth in the old and new steady-state.

\item  If the steady-state aggregate wealth is higher in the original policy regime than in the new policy-regime, then the agents will spend the transition periods gradually ``spending down'' their wealth.  Wealth gradually decreases from its old to new steady-state value.  Aggregate consumption, leisure, driving, and emissions gradually decrease during the transition.  Thus, when aggregate consumption, aggregate leisure, aggregate driving, and aggregate emissions jump at the time when the new policy-regime is announced, they all jump to values higher than their eventual new steady-state values, which gives them room to decline.  Similarly, aggregate labor jumps to a value below its eventual steady-state value, which gives it room to gradually increase during the transition.

\item If the steady-state aggregate wealth is lower in the original policy regime than in the new policy-regime, the agents ``save up'' their wealth during the transition.  Wealth gradually increases.  Aggregate consumption, leisure, driving, and emissions jump to values below their eventual new steady-state values so that they have room to increase, and vice-versa for labor.

\item If the steady-state aggregate wealth is about the same in the original policy regime as in the new policy-regime, then aggregate consumption, leisure, driving, emissions, and labor all jump to their new steady-states, with no periods of transition.

\item  Which policy experiments lead to higher or lower wealth?  The transition from baseline to having a wage tax (with or without a rebate) lowered the steady-state value of wealth.  The transition from baseline to having an emissions tax (no rebate) led to no change in wealth.  The transition from baseline to having an emission tax with rebate, or to a mandatory retirement age, led to an increase in wealth.  

\item  The transition from baseline to having a mandated work limit leads to wealth being gradually spent down, but not quite in the way described above.  Consumption, driving, and pollution jump to higher than their eventual steady-state values, so they have room to gradually decline.  However, labor jumps immediately to its new steady-state value of 0.4, which is the mandated limit, and leisure correspondingly jumps almost immediately to its new steady-state value, since it is defined as time not driving or doing labor.

%\item In these examples, the transition between steady-states usually takes about 13 to 15 periods (52 to 60 years).  

\item  The transitions are generally characterized by aggregate labor and aggregate emissions moving in opposite directions.  To explain why this is, let us consider the case where the steady-state value of wealth is higher in the original steady-state than in the new steady-state.  During the transition, aggregate consumption, aggregate leisure, and aggregate driving are all higher than they will eventually be in the new steady-state.  This is because agents are ``spending down'' their wealth during the transition to get to the new, lower value of wealth.  Since the agents in the model are spending down their wealth, they are consuming more, which means foreigners (that is, agents outside the model) are performing labor to allow them to consume and drive.  (Similarly, when agents in this model build up wealth in the aggregate, it means foreigners are paying them for the right to consume some of the fruits of their labor.)  

%\item In this model, I have defined pollution emissions to be emitted by agents that consume and drive, and by governments that spend.  The emissions - as I have defined them - are not emitted by the labor which produces the goods and services that are consumed.


\end{enumerate}


%\unappendix





\end{document}





